\chapter{Estado del arte}
\label{chapter:related-work}

Este capítulo se enfoca en mostrar los diferentes trabajaos que permitirán
comprender mejor la problemática que se quiere resolver, así como las
diferentes propuestas para poder resolverlo.

\section{Competencia IEEE CoG}
\label{section:ieeecog}

Como se demuestra en el artículo presentado por J. Renz et al.\cite{Renz2016}, la
competencia de Angry Birds utilizando inteligencia artificial se realiza de
manera anual desde el año 2012 iniciando con la conferencia dentro del marco de
la AAAI (Association for the Advancement of Artificial Intelligence), en este
artículo se explica la primera línea de competencia que consta de desarrollar
agentes que sean capaces de solucionar niveles del juego de tal manera que se
asemeje a la manera en cómo jugaría una persona real, se explica el estado del
arte el cual consta de previos competidores y de cómo han logrado realizar sus
agentes para el juego, de igual manera se explica la nueva línea de competencia
que consta de utilizar el mismo conjunto de niveles y permitir que los agentes
compitan para ver cual logra resolver más rápido el conjunto de niveles.

Posteriormente en un artículo elaborado por M. Stephenson et
al.\cite{Stephenson2018The2A} se explica detalladamente como es que el juego se
conecta en una arquitectura cliente-servidor y como es que se adapta la
arquitectura para que los agentes utilizados puedan obtener datos del juego para
poder realizar los cálculos pertinentes, en este mismo artículo se presenta
información de la segunda línea de competencia llamada competencia de generación
de niveles de AIBIRDS la cual se comenzó a realizar desde el año 2016, en esta
versión de la competencia los generadores deben de cumplir con el propósito de
crear niveles que sean creativos, divertidos, estables en cuanto a la gravedad
del juego y posibles de ser solucionados, además de esto se debe de tener en
consideración realizar niveles que sean desafiantes a los jugadores, como el
juego no es open-source los niveles se generan en una versión clon del juego que
cuenta con las mismas mecánicas.

Un segundo artículo escrito por M. Stephenson et al.\cite{Stephenson2018} se
presenta una versión más actualizada de las reglas de la competencia, en este
artículo se establece el uso de un archivo proporcionado por los organizadores,
este archivo cuenta con 4 líneas en las cuales se establece las condiciones que
los niveles generados deberán de cumplir, la información proporciona en el
archivo es la cantidad de niveles que se deberán de generar, las piezas no
permitidas en un nivel, así como las combinaciones que se deberán de evitar, la
cantidad de puercos a colocar en un nivel y por último el tiempo límite para
terminar los niveles, en el caso de los tipos de materiales prohibidos se
entrega una lista que especifica que piezas con que material no se puede
colocar, en el caso de los puercos se entrega un rango numérico en el cual se
especifica la cantidad mínima y máxima a colocar, en cuanto a los valores de
niveles y tiempo límite se entregan como valores numéricos enteros, sin embargo
los valores que se entregan generalmente son 5 para la cantidad de niveles y 30
en la cantidad de minutos que puede durar el generador.

\section{Algoritmos de búsqueda}
\label{section:search-based}

L. Ferreira et al.\cite{Ferreira2014} presentaron un sistema de generación de
niveles mediante el uso de algoritmos genéticos, en el sistema se propone la
definición de un individuo como una lista de elementos que se acomodaran en
forma de torre, el sistema propuesto utiliza las piezas base del juego y se
agregaron 4 elementos compuestos, los niveles se definen como un área de trabajo
con tres posiciones donde las torres podrán ser generadas a base de las piezas
en la lista principal, las piezas irregulares tales como los dos diferentes
tipos de triángulos son tomados en cuenta únicamente en la última posición de
las torres.

Una herramienta de generación de niveles basado en el uso de patrones de
generación es propuesto por Y. Jiang et al.\cite{Jiang2017}, en este trabajo los
autores proponen el uso de patrones predefinidos de generación de estructuras
basadas en el alfabeto americano, valores numéricos, símbolos, así como el
ordenamiento de estos en base a patrones de frases o palabras predefinidas que
pueden ser utilizadas, utilizando estos patrones se generan niveles en donde se
muestran frases divertidas o motivacionales que serían jugados por personas para
evaluarlos, debido a la manera de acomodo de bloques utilizada en este paper el
problema de que los niveles sean estables se eliminaba y simplemente se utiliza
el sistema para cumplir los dos objetivos restantes los cuales son que los
niveles sean jugables y entretenidos, mientras que la manera final de evaluar la
funcionalidad del generador se basaba la re-jugabilidad, la legibilidad del texto
así como de la dificultad de los niveles, para esto los autores se apoyaron de
10 personas con cierto grado de comprensión del idioma inglés, de esta manera en
caso de que los textos fueran difíciles de comprender, la dificultad fuera
demasiado alta o generara un interés por repetir el nivel los participantes
serían los encargados de proporcionar estos resultados. % Agrega imágenes de los niveles.

Para la adaptabilidad de los niveles basados en los resultados de la función de
aptitud M. Kaidan et al.\cite{Kaidan2015} proponen una medición basada en la
habilidad de un jugador en un nivel, la evaluación de la habilidad está dada por
el puntaje obtenido durante el juego, en el juego de Angry Birds el puntaje de
un jugador está ligado a la cantidad de destrucción de estructuras lograda y la
cantidad de aves que no se utilizaron para completar el nivel, en este caso el
paper se apoyaba de la función de aptitud de L. Ferreira et
al.\cite{Ferreira2014} para la distribución de los puercos en un nivel y se
adaptaba la función para modificar la cantidad de puercos colocados de acuerdo a
la facilidad con la que se lograba resolver el nivel, de esta manera el sistema
generaba un conjunto de niveles para que una persona los solucionara y basándose
en los resultados de los niveles se generaba el siguiente conjunto para
resolver, de tal manera que el sistema requería que una persona evaluara los
niveles cada generación para que el sistema generara niveles más desafiantes
para esa persona.

Otro generador propone el uso de un selector de dificultad como parámetro de
entrada de un algoritmo genético para la generación de niveles con elementos
predefinidos, en este paper M. Kaidan et al.\cite{Kaidan2016} utilizan una
versión extendida del Algebra de Intervalo Temporal de Allen (Allen's Temporal
Interval Algebra)\cite{ALLEN1990} para calcular área mínima delimitadora
(Minimum Bounding Rectangle) de cada elemento del juego para realizar cálculos
de posición más exactos al momento de obtener resultados, utilizando estas
mecánicas proponen evaluar la relación de movimiento después de que un ave a
impactado en alguna de las piezas del nivel.

Otros investigadores proponen utilizar sistemas de generación que no incluyan
limitantes en las estructuras a fin de permitir una evolución general mas
"libre" tal es el caso del paper desarrollado por L. Calle et al.
\cite{Calle2019} en donde utilizan un sistema de generación que ignora la
necesidad de crear estructuras siguiendo patrones o simetría, en este paper se
busca utilizar un espacio de búsqueda relativamente pequeño para ahorrar tiempo
de procesamiento, además de esto utilizan una aplicación de código libre llamada
Box2D (https://box2d.org) sobre la cual está basado el código del juego de Angry
Birds y permite realizar simulaciones de manera más rápida debido a que solo se
calcula la estabilidad de las estructuras para el valor de fitness de los
individuos, sin embargo, debido que el sistema de generación no está encaminado
por patrones las estructuras finales generadas carecen de llamatividad visual,
cabe mencionar que este trabajo se utilizó como base para el desarrollo del
proyecto actual, algunas de las ideas tuvieron origen en este trabajo para ser
modificadas y adaptadas en nuestra propuesta.

\section{Agentes para jugar}
\label{section:others}

De igual manera varios autores toman la ruta de programación de agentes que sean
capaces de solucionar los niveles presentados en el juego, de estos se hacen
mención algunos.

Dentro de la competencia de diseño de agentes el objetivo es que el agente sea
capaz de solucionar los niveles utilizando las mecánicas de gravedad y colisión
del juego, por tal manera F. Calimeri et al. \cite{Calimeri2016} propones el uso
de una forma de programación declarativa llamada Answer Set Programming (ASP por
sus siglas en inglés) dicha mecánica está orientada hacia algoritmos de búsqueda
difíciles, en este caso utilizan ASP para analizar el campo de juego que
involucra los bloques utilizados en las estructuras y los puercos estacionados
en ellas, de esta manera ASP tratara de identificar el daño que puede ser
causado, así como el movimiento resultante de los bloques en base a la gravedad
del juego al momento de disparar un ave en un cierto angulo con determinada
fuerza, en base a esto se determina el siguiente tiro a realizar para poder
completar el nivel. 

Diferentes autores han utilizado diferentes técnicas para la generación de
contenido, algunas utilizando computación evolutiva y algunas otras simplemente
a base de algoritmos de búsqueda, de estos mismos los más relevantes se les hace 
% Solo mencionas uno 
mención: G. Smith et al.\cite{Smith2009} proponen un software de generación de
niveles estilo juego de plataformas mediante el uso de un software aplicando
inteligencia artificial, en este software se le permite a un usuario seleccionar
la altura del punto de inicio y punto final de un nivel en particular y el
sistema trata de llenar el espacio restante con plataformas que permitan que el
nivel se pueda completar, el sistema se basa en una medición de pulsos en los
cuales una persona podría realizar una acción determinada para avanzar a la
siguiente sección.
