\chapter{Estado del arte}
\label{chapter:related-work}

Este capitulo se enfoca en mostrar los diferentes trabajaos que permitiran
comprender mejor la problematica que se quiere resolver asi como los diferentes
enfoques que se le han dado para poder resolverlo.

\section{Competencia IEEE CoG}
\label{section:ieeecog}

Como se demuestra en el articulo presentado por J. Renz et al.\cite{Renz2016} la
competencia de Angry Birds utilizando inteligencia artificial se realiza de
manera anual desde el año 2012 iniciando en la conferencia dentro del marco de
la AAAI (Association for the Advancement of Artificial Intelligence), en este
articulo se explica la primera linea de competencia que consta de desarrollar
agentes que sean capaces de solucionar niveles del juego de tal manera que se
asemeje a la manera en como jugaria una persona real, se explica el estado del
arte el cual consta de previos competidores y de como han logrado realizar sus
agentes para el juego, de igual manera se explica la nueva linea de competencia
que consta de utilizar el mismo conjunto de niveles y permitir que los agentes
compitan para ver cual logra resolver mas rapido el conjunto de niveles.

Posteriormente en un articulo elaborado por M. Stephenson et
al.\cite{Stephenson2018The2A} se explica detalladamente como es que el juego se
conecta en una arquitectura cliente-servidor y como es que se adapta la
arquitectura para que los agentes utilizados puedan obtener datos del juego para
poder realizar los calculos pertinentes, en este mismo articulo se presenta
informacion de la segunda linea de competencia llamada competencia de generacion
de niveles de AIBIRDS la cual se comenzo a realizar desde el año 2016, en esta
version de la competencia los generadores deben de cumplir con el proposito de
crear noveles que sean creativos, divertidos, estables en cuanto a la gravedad
del juego y posibles de ser solucionados, ademas de esto se debe de tener en
consideracion realizar niveles que sean desafiantes a los jugadores, como el
juego no es open source los niveles se generan en una version clon del juego que
cuenta con las mismas mecanicas.

Un segundo articulo escrito por M. Stephenson et al.\cite{Stephenson2018} se
presenta una version mas actualizada de las reglas de la competencia, en este
articulo se establece el uso de un archivo proporcionado por los organizadores,
este archivo cuenta con 4 lines en las cuales se establece las condiciones que
los niveles generados deberan de cumplir, la informacion proporciona en el
archivo es la cantidad de niveles que se deberan de generar, las piezas no
permitidas en un nivel asi como las combinaciones que se deberan de evitar, la
cantidad de puercos a colocar en un nivel y por ultimo el tiempo limite para
terminar los niveles, en el caso de los tipos de materiales prohibidos se
entrega una lista que especifica que piezas con que material no se puede
colocar, en el caso de los puercos se entrega un rango numerico en el cual se
especifica la cantidad minima y maxima a colocar, en cuanto a los valores de
niveles y tiempo limite se entragan como valores numericos enteros, sin embargo
los valores que se entregan generalmente son 5 para la cantidad de niveles y 30
en la cantidad de minutos que puede durar el generador.

\section{Algoritmos de busqueda}
\label{section:search-based}

L. Ferreira et al.\cite{Ferreira2014} presentaron un sistema de generación de
niveles mediante el uso de algoritmos genéticos, en el sistema se propone la
definición de un individuo como una listas de elementos que se acomodaran en
forma de torre, el sistema propuesto utiliza las piezas base del juego y se
agregaron 4 elementos compuestos, los niveles se definen como un área de trabajo
con tres posiciones donde las torres podrán ser generadas a base de las piezas
en la lista principal, las piezas irregulares tales como los dos diferentes
tipos de triángulos son tomados en cuenta únicamente en la última posición de
las torres.

Una herramienta de generacion de niveles basado en el uso de patrones de
generacion es propuesto por Y. Jiang et al.\cite{Jiang2017}, en este paper los
autores propoen el uso de patrones predefinidos de generacion de estructuras
basadas en el alfabeto americano, valores numericos, simbolos asi como el
ordenamiento de estos en base a patrones de frases o palabras predefinidas que
pueden ser utilizadas, utilizando estos patrones se generan niveles en donde se
muestran frases divertidas o motivacionales que serian jugados por personas para
evaluarlos, debido a la manera de acomodo de bloques utilizada en este paper el
problema de que los niveles sean estables se eliminaba y simplemente se utiliza
el sistema para cumplir los dos objetivos restantes los cuales son que los
niveles sean jugables y entretenidos, mientras que la manera final de evaluar la
funcionalidad del generador se basaba la rejugabilidad, la legibilidad del texto
asi como de la dificultad de los niveles, para esto los autores se apoyaron de
10 personas con cirto grado de comprension del idioma ingles, de esta manera en
caso de que los textos fueran dificiles de comprender, la dificultad fuera
demasiado alta o generara un interes por repetir el nivel los participantes
serian los encargados de proporcionar estos resultados.

Para la adaptabilidad de los niveles basados en los resultados de la funcion de
aptitud M. Kaidan et al.\cite{Kaidan2015} proponen una medicion basada en la
habilidad de un jugador en un nivel, la evaluacion de la habilidad esta dada por
el puntaja obtenido durante el juego, en el juego de angry birds el puntaje de
un jugador esta ligado a la cantidad de destruccion de estructuras lograda y la
cantidad de aves que no se utilizaron para completar el nivel, en este caso el
paper se apoyaba de la funcion de aptitud de L. Ferreira et
al.\cite{Ferreira2014} para la distribucion de los puercos en un nivel y se
adaptaba la funcion para modificar la cantidad de puercos colocados deacuerdo a
la facilidad con la que se lograba resolver el nivel, de esta manera el sistema
generaba un conjunto de niveles para que una persona los solucionara y basandose
en los resultados de los niveles se generaba el siguiente conjunto para
resolver, de tal manera que el sistema requeria que una persona evaluara los
niveles cada generacion para que el sistema generara niveles mas desafiantes
para esa persona.

Otro generador propone el uso de un selector de dificultad como parametro de
entrada de un algoritmo genetico para la generacion de niveles con elementos
predefinidos, en este paper M. Kaidan et al.\cite{Kaidan2016} utilizan una
version extendida del Algebra de Intevalo Temporal de Allen (Allen's Temporal
Interval Algebra)\cite{ALLEN1990} para calcular area minima delimitadora
(Minimum Bounding Rectangle) de cada elemento del juego para realizar calculos
de posicion mas exactos al momento de obtener resultados, utilizando estas
mecanicas proponen evaluar la relacion de movimiento despues de que una ave a
impactado en alguna de las piezas del nivel.

Otros investigadores proponen utilizar sistemas de generacion que no incluyan
limitantes en las estructuras a fin de permitir una evolucion general mas
"libre" tal es el caso del paper desarrollado por L. Calle et al.
\cite{Calle2019} en donde utilizan un sistema de generacion que ignora la
necesidad de crear estructuras siguiendo patrones o simetria, en este paper se
busca utilizar un espacio de busqueda relativamente pequeño para ahorrar tiempo
de procesamiento, ademas de esto utilizan una aplicacion de codigo libre llamada
Box2D (https://box2d.org) sobre la cual esta basado el codigo del juego de Angry
Birds y permite realizar simulaciones de manera mas rapida debido a que solo se
calcula la estabilidad de las estructuras para el valor de fitness de los
individuos, sin embargo debido que el sistema de generacion no esta encaminado
por patrones las estructuras finales generadas carecen de llamatividad visual,
cabe mencionar que este trabajo se utilizo como base para el desarrollo del
proyecto actual, algunas de las ideas tuvieron origen en este trabajo para ser
modificadas y adaptadas en nuestra propuesta.

\section{Agentes para jugar}
\label{section:others}

De igual manera varios autores toman la ruta de programacion de agentes que sean
capaces de solucionar los niveles presentados en el juego, de estos se hacen
mencion algunos.

Dentro de la competencia de diseño de agentes el objetivo es que el agente sea
capaz de solucionar los niveles utilizando las mecanicas de gravedad y colision
del juego, por tal manera F. Calimeri et al. \cite{Calimeri2016} propones el uso
de una forma de programacion declarativa llamada Answer Set Programming (ASP por
sus siglas en ingles) dicha mecanica esta orientada hacia algoritmos de busqueda
dificiles, en este caso utilizan ASP para analizar el campo de juego que
involucra los bloques utilizados en las estructuras y los puercos estacionados
en ellas, de esta manera ASP tratara de identificar el daño que puede ser
causado asi como el movimiento resultante de los bloques en base a la gravedad
del juego al momento de disparar una ave en un cierto angulo con determinada
fuerza, en base a esto se determina el siguiente tiro a realizar para poder
completar el nivel. 

Diferentes autores han utilizado diferentes técnicas para la generación de
contenido, algunas utilizando computación evolutiva y algunas otras simplemente
a base de algoritmos de búsqueda, de estos mismos los mas relevantes se les hace
mención: G. Smith et al.\cite{Smith2009} proponen un software de generación de
niveles estilo juego de plataformas mediante el uso de un software aplicando
inteligencia artificial, en este software se le permite a un usuario seleccionar
la altura de el punto de inicio y punto final de un nivel en particular y el
sistema trata de llenar el espacio restante con plataformas que permitan que el
nivel se pueda completar, el sistema se basa en una medición de pulsos en los
cuales una persona podría realizar una acción determinada para avanzar a la
siguiente sección.


