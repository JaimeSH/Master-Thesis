% $Log: abstract.tex,v $
% Revision 1.1  93/05/14  14:56:25  starflt
% Initial revision
% 
% Revision 1.1  90/05/04  10:41:01  lwvanels
% Initial revision
% 
%
%% The text of your abstract and nothing else (other than comments) goes here.
%% It will be single-spaced and the rest of the text that is supposed to go on
%% the abstract page will be generated by the abstractpage environment.  This
%% file should be \input (not \include 'd) from cover.tex 

En el documento de tesis presentado se propone un algoritmo evolutivo que
utiliza enfoques de la teoría de open-ended evolution para generar niveles para
el videojuego de Angry Birds. Los niveles utilizados en este juego se componen
de elementos tales como una cierta cantidad de aves que el usuario puede
disparar a manera de resortera, una cantidad determinada de puerquitos que se
requieren eliminar como objetivo para avanzar en los niveles, así como de una
determinada cantidad de piezas que conforman estructuras que sirven como
obstáculos para el usuario. El objetivo del sistema que aquí se propone es el de
generar las diferentes estructuras que se presentan en el juego, teniendo en
cuenta las características que deben de cumplir para poder ser utilizadas, es
decir que sean llamativas y funcionales. Para esto se propone una búsqueda de
múltiples capas, la primera parte se trata de construir estructuras utilizando
piezas básicas del juego, posteriormente utilizar estas estructuras como base y
continuar generando desde ese punto para obtener estructuras más complejas. Se
utilizará un enfoque basado en open-ended evolution el cual será utilizado para
la evolución de las estructuras con el objetivo de crear otras más complejas y
diversas entre sí. La función de aptitud que se utilizara para evaluar los
niveles considera la complejidad de los niveles y que tan diferentes son estos
del resto de la población de niveles generada. Los experimentos realizados
muestran que un enfoque evolutivo permite la generación de niveles que logran
ser novedosos y son interesantes para ser jugados.
 
\clearpage
\section*{Abstract}

In this thesis, we propose an evolutionary algorithm that follows an open-ended
evolution approach to generate levels for the Angry Birds video game. The levels
themselves are composed of a set of birds that the player can throw with a
slingshot, a certain amount of pigs that they must destroy, and a given number
of pieces that conform structures that may or may not protect the pigs from
birds thrown at them. The current goal is the generation of diverse structures
that are playable in the game, having the additional characteristics of being
fun and enjoyable. We propose a multi-layered search, first constructing
composite structures from basic blocks, to then build more complex structures
building from these composites. We follow an open-ended evolution approach in
which the evolution of structures is not guided towards a single objective but
is rather free to evolve and generate novelty or diversity. The fitness function
we use to evaluate the proposed levels considers how complex levels have become
and how different they are from the rest of the population. The experiments
conducted show that an evolutionary approach allows the generation of levels
that are novel and interesting to play.