% $Log: abstract.tex,v $
% Revision 1.1  93/05/14  14:56:25  starflt
% Initial revision
% 
% Revision 1.1  90/05/04  10:41:01  lwvanels
% Initial revision
% 
%
%% The text of your abstract and nothing else (other than comments) goes here.
%% It will be single-spaced and the rest of the text that is supposed to go on
%% the abstract page will be generated by the abstractpage environment.  This
%% file should be \input (not \include 'd) from cover.tex.

El entender el comportamiento de los mercados financieros es importante para
determinar el estado de la economía de un país. Las herramientas actuales para
entender estos comportamientos son variadas y usualmente pueden ser asignadas a
una de dos categorías: análisis fundamentalista o técnico. El análisis
fundamentalista se basa en analizar cualquier factor que pueda afectar el valor
de un activo, tales como condiciones financieras o la administración de una
compañía, mientras que el análisis técnico se enfoca exclusivamente en analizar
los movimientos de los precios de un mercado. Esta tesis presenta un método de
predicción para mercados financieros basado en análisis técnico. El método puede
ser usado como una estrategia de intercambio y como una herramienta para
entender el comportamiento de un mercado financiero y sigue una arquitectura
basada en agentes donde las reglas de los agentes están basadas en lógica
difusa. Específicamente se usan sistemas difusos intuicionistas que le permiten
a los agentes modelar tanto la incertidumbre como la indecisión en un mercado.
Para poder modelar la percepción de los agentes, los precios de un mercado son
preprocesados usando un algoritmo basado en un indicador técnico basado en
retrasos. Una implementación del método propuesto demuestra la versatilidad del
sistema, ya que los modelos generados pueden llevar a estrategias de intercambio
rentables, así como modelos que pueden ser interpretados por un ser humano para
obtener posibles explicaciones sobre el comportamiento de un mercado.

\clearpage
\section*{Abstract}

In this paper, we propose an evolutionary algorithm that follows an open-ended
evolution approach to generate levels for the Angry Birds video game. The levels
themselves are composed of a set of birds that the player can throw with a
slingshot, a certain amount of pigs that they must destroy, and a given number
of pieces that conform structures that may or may not protect the pigs from
birds thrown at them. The current goal is the generation of diverse structures
that are playable in the game, having the additional characteristics of being
fun and enjoyable. We propose a multi-layered search, first constructing
composite structures from basic blocks, to then build more complex structures
building from these composites. We follow an open-ended evolution approach in
which the evolution of structures is not guided towards a single objective but
is rather free to evolve and generate novelty or diversity. The fitness function
we use to evaluate the proposed levels considers how complex levels have become
and how different they are from the rest of the population. The experiments
conducted show that an evolutionary approach allows the generation of levels
that are novel and interesting to play.