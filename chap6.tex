\chapter{Experimentos y Resultados}
\label{chapter:experiments-and-results}

Este capitulo es el encargado de describir los diferentes experimentos
realizados con el sistema, dichos experimentos se realizaron para comprobar la
correcta funcionalidad de lo creado asi como para revisar que efectivamente el
sistema hace lo que fue propuesto en capitulos anteriores, eso es la generacion
de niveles estables y diferentes entre si.

La manera en como se representaran estos resultados sera mediante el uso de dos
graficas, la primera establece los valores de fitness minimos y maximos que se
logran alcanzar durante las generaciones, ademas de esto en la misma grafica se
presenta una linea extra que representa el promedio de los fitness de los mismos
individuos, siguiente de esto se presenta una segunda grafica que representa la
\textit{distancia hamming} minima alcanzada en cada conjunto de individuos
durante las generaciones, como se explico anteriormente esta grafica representa
el valor minimo de cambios entre los conjuntos de listas de genotipos de los
individuos, esta grafica tendra la tendencia de ir reduciendose conforme avanzan
las generaciones debido a que mientras mas se avanza en las generaciones los
elementos elite tenderan a ir apareciendo mas.

La manera en como se define el valor de fitness de los individuos es como se
explica en la seccion \ref{subsection:fitness_calculation} es mediante la
separacion de los dos aspectos de importancia en los niveles, primero la
\textit{estabilidad} que define el comportamiento del genotipo durante la
simulacion y la segunda siendo \textit{diversidad} que define lo \textit{nuevo}
que logran ser los niveles generados.

Los parametros del algoritmo genetico explicados en el capitulo
\ref{chapter:implementation} se muestran en la tabla \ref{table:parametros_ga}.

\begin{table}[ht]
  \caption{Parametros utilizados en el algoritmo genetico}
  \label{table:parametros_ga}
  \centering
  \begin{tabular}{|c|c|}
  \hline
  Parametro & Valor \\
  \hline
  \hline
  Fitness Function & Estabilidad \\ & Diversidad \\
  \hline
  Tamaño de la poblacion & 10 o 20 \\
  \hline
  Numero de Generaciones & 100 o 10 (respectivamente) \\
  \hline
  Criterio de parada & Numero de Generaciones \\
  \hline
  Operador de seleccion & Seleccion por torneo \\
  \hline
  Operador de cruce & Cruce de un punto \\
  \hline
  Porcentaje de cruce & 30\% \\
  \hline
  Operador de mutacion & Mutacion de individuo \\ & Mutacion de compuestos \\ & Mutacion de material \\
  \hline
  Poercentaje de mutacion & 30\% \\
  \hline
  \end{tabular}
\end{table}


El contenido de este capitulo se encargara de mostrar las capacidades de
generacion del sistema propuesto e implementado segun lo mostrado en el capitulo
\ref{chapter:implementation} deacuerdo a las areas explicadas anteriormente.
Cada experimento mostrado se describira en los ambitos de la capacidad de
generar niveles \textit{estables} asi como tambien la capacidad de generar
niveles \textit{diversos}, puesto que se tienen una gran cantidad de indivudos
en las simulaciones entonces para demostrar ambos ambitos se tomaran el
\textit{mejor} y el \textit{peor} del final de cada experimento asi como un
indiviudo aleatorio de la primera generacion, utilizando estos niveles generados
se explicara la manera en como se evoluciono la diversidad de los niveles y
cuales fueron los compuestos que se pueden rescatar de las simulaciones para ser
utilizados en experimentos subsecuentes, para tener un campo nivelado para todos
los experimentos aquellos compuestos que aparecen durante los experimentos no
son reutilizados en otros experimentos subsecuentes.

Todos los experimentos mostrados en esta seccion fueron optimizados durante un
total de 10 generaciones, esto es debido principalmente a que para realizar las
simulaciones se requiere una gran cantidad de tiempo, esto sumado con la
cantidad diferente de indiviudos, en experimentos de 100 generaciones se
utilizaban un total de 10 individuos y para experimentos de 10 generaciones se
utilizo un total de 20, ademas de esto en muchos de los experimentos de 100
generaciones el algoritmo llegaba a un punto de estancamiento generalmente antes
de las primeras 50 generaciones, por tal motivo se decidio reducir la cantidad
de generaciones a 10 pero incrmementar el total de indiviudos para obtener una
mayor diversidad.

Como se explico en el parrafo anterior los experimentos que se presentaran
cubriran las variables de 20 individuos durante un total de 10 generaciones y
utilizando las variables presentadas en la tabla \ref{table:parametros_ga}.
Utilizando estas configuraciones se realizaron un total de [Write number here?]
sin embargo debido a que los resultados mostrados por cada experimentos son
demasiados solo algunos resultados de experimentos seran mostrados.

\section{Generacion de niveles estables}
\label{section:chap6_stable_level_generation}

La primera seccion de resultados se enfoca a mostrar lo que el sistema es capaz
de generar en caunto a niveles estables se refiere, para este aspecto se tomaron
en cuenta los puntos propuestos en la seccion \ref{chapter:proposed-method} en
donde se busca obtener los mejores niveles tomando en cuenta las posiciones de
las piezas colocadas antes y despues de las simulaciones siempre teniendo en
cuenta que aquellos niveles que logren mantenerse de pie o inclusive mantener la
mayor cantidad de compuestos aun presentes en el nivel son los que se consideran
como los mejores niveles para cada generacion, de la misma manera en como se
explico anterioremente aquellos niveles que logren cumplir estas caracteristicas
son registrados en una lista que mantiene la informacion de los mejores niveles
generados a lo largo de las generaciones y elementos de este mismo grupo son
reintegrados a la poblacion al iniciar la siguiente generacion.

La manera en como se representan los resultados de las simulaciones es mediante
el uso de 2 graficas que muestran los valores de la funcion de aptitud
obtenidos, en esta grafica se muestra el promedio de los individuos en una
generacion dada, el valor minimo que se alcanza en dicha generacion asi como el
valor de aptitud maximo que se logra obtener en las generaciones, en este
aspecto cabe resaltar que obtener no todos los individuos con un valor de 100\%
de aptitud seran los mejores elementos de la pobalcion, sino que simplemente la
manera en como se generaron los niveles no permite que los compuestos sean
destruidos para esto es necesario revisar de manera visual los resultados y
darse cuenta de que elementos que logran obtener valores diferentes al 100\% de
aptitud pueden ser elemntos validos si las estructuras generadas logran cumplir
con los objetivos buscados, despues de la grafica de aptitud de presentan dos
ejemplos de los niveles generados, aquellos denotados por b) representan el
"mejor" individuo de la generacion, mientras que aquellos representados por c)
representan el "peor" elemento de la primera generacion, estsos ejemplos son
representados a su vez de dos maneras, la imagen superior representa el estado
inicial del individuo, es decir al momento de iniciar la simulacion, mientras
que la imagen inferior representa el estado final del mismo individuo, es decir
el punto en el que la simulacion termino, mediante el uso de estas imagenes es
posible apreciar la manera en como los niveles generados se comportan en
terminos de estabilidad bajo la gravedad proporcionada en el juego.

De igual manera se presenta informacion de la ultima generacion de la
simulacion, en este caso se toman unicamente los dos indiviudos (diferentes
entre si) que representan los mejores individuos de la poblacion, en este caso
se toman dos individuos diferentes debido a que debido a la naturaleza de los
algoritmos geneticos la tendencia de los individuos es asemejarse unos a otros y
mediante esto se busca mostrar que es posible inclusive en la generacion de
niveles obtener elementos que no necesariamente son los mejores generados pero
logran acercarse lo suficiente a los objetivos del sistema.

\subsection{Experimento 1}
\label{chap6:exp_1}

La figura \ref{figure:exp_01_a} mas especificamente en la parte
\ref{fig:exp1_first} la grafica de los resultados de aptitud obtenidos durante
las simulaciones, en esta grafica se puede notar que durante las primeras
generaciones se un elemento de la poblacion siempre mantiene el mismo nivel de
estabilidad, una vista de este elemento de la poblacion se puede aprecia en la
sub figura \ref{fig:exp1_third} de la misma imagen, asi mismo se obtuvo uno de
los "peores" elementos de la generacion el cual se muestra en la sub figura
\ref{fig:exp1_fourth}, como se puede apreciar en la imagen este elemento de la
poblacion inicia la simulacion con varias piezas muy elevadas e inclusive mal
acomodadas lo cual genera que al terminar la simulacion solo una parte de las
estructuras originales se mantenga de pie.

Como es posible apreciar en la grafica este conjunto de niveles con dicha
estabilidad se logra mantener durante varias generaciones no es sino hasta las
ultimas dos generaciones en donde por medio de las operaciones de cruce y
mutacion se obtiene un nuevo indivuo capaz de llegar al 80\% de estabilidad,
este nuevo elemento elemento se puede aprecia en \ref{fig:exp1_fifth}, aqui
mismo se puede aprecia que a pesar de tenener el mejor resultado de aptitud de
la generacion en ocasiones no es posible mantener el balance en todos los
elementos, sin embargo muchos se logran mantener sin ser destruidos, de igual
manera se presenta en \ref{fig:exp1_sixth} otro individuo representante de la
ultima generacion de la simulacion, en este caso este segundo individuos
representa el segundo mejor valor de aptitud que no es completamente igual a
aquel que esta en la primera posicion, esto es debido a que al integrar miembros
de elite de vuelta a la poblacion muchas veces se integran mas de uno que son
invididuos completamente iguales debido a que si durante una generacion dada el
valor de aptitud no fue mejor que el de un miembro de elite ya existente
entonces el mejor de la generacion muchas veces siendo el mismo que se reintegro
a la poblacion es el que se agrega de vuelta a la lista de miembros elite.

\import{extras/experimentos/exp_01/}{test.tex} 
%\subimport{extras/experimentos/exp_01/}{test.tex}

\newpage

\subsection{Experimento 2}
\label{chap6:exp_2}

Como se puede apreciar en la figura \ref{figure:exp_03_a} o mas directamente en
\ref{fig:exp3_first} esta experimento inicio mal debido a la manera en como se
generaron los compuestos al inciar las generaciones, de igual manera este inicio
no tan bueno se puede aprecia en la sub figura \ref{fig:exp3_third} en donde
varios de los elementos generados no logran mantenerse en buen balace y terminan
cayendo, algunos destruyendose en el proceso, de igual manera en la sub figura
\ref{fig:exp3_fourth} del peor individuo de la generacion se puede ver que
muchos de los elementos inician su posicion en posiciones muy altas o en el caso
mostrado en esta figura con acomodos muy extraños que terminan por destruir toda
la estructura que se esperaba crear en ese punto.

A pesar de tener mal acomodo y distribucion de los elementos en el caso del peor
individuo, se puede apreciar que a pesar de generar estructuras mal ordenadas es
posible generar estructurs que por la manera en como se acomodan debido a la
gravedad son capaces de mantenerce de pie, este tipo de estructuras aun siendo
visualmente un acomodo "errado" de las piezas puede ser considerado como una
estructura que puede ser rescatada para su uso en futuras generaciones.

A diferencia del primer experimento presentado en esta seccion este es capaz de
comenzar a subir el nivel de aptitud a partir de la segunda generacion de
individuos, a pesar de ser capaz de subir la aptitud de la primera a la segunda
generacion despues de eso el sistema no es capaz de continuar mejorando estos
resultados en este caso los resultados se quedan estancados en niveles como los
que se muestran en la sub figura \ref{fig:exp3_fifth} la cual a pesar de lograr
mantener la mayor cantidad de elementos durante la simulacion tiene el problema
de que las piezas inician en posiciones muy elevadas y al momento de ser movidas
por la gravedad pierden su posicion original lo cual provoca que el valor de
aptitud se vea afectado, el mejor individuo comparte el mismo problema con el
segundo mejor mostrado en la sub figura \ref{fig:exp3_sixth} donde se puede
apreciar que las extructuras a pesar de estar aun de pie dado el tiempo
necesario estas estructuras terminaran cayendo hacia un lado u otro, a pesar de
esto este tipo de estructuras generadas permiten ver que existen compuestos en
donde a pesar de mantener los elementos de pie la base de alguna de las
estructuras tiene el comportamiento de una balanza debido al uso de estructuras
triangulares en las cuales un pequeño cambio de peso en alguno de los lados del
mismo provocara que la estructura completa termine cayendo, sin embargo este
tipo de estructuras proveeen un punto de interes debido a lo visualmente
complejo que pueden ser este tipo de estructuras.

\import{extras/experimentos/exp_03/}{test.tex}

\newpage

\subsection{Experimento 3}
\label{chap6:exp_3}

Un tercer grupo de resultados de experimentos se presenta en la figura
\ref{figure:exp_04_a} mas directamente mediante la grafica de
\ref{fig:exp4_first} este experimento en particular se puede aprecia tuvo un
inicio no muy bueno iniciando en aproximadamente 40\% de aptitud de estabilidad,
esto se puede apreciar mas detalladamente en la imagen \ref{fig:exp4_third}
donde se puede ver que el mejor individuo de la primera generacion inicia muy
mal la simulacion al tener elementos que inicial con muy mal acomodo lo cual
provoca que la estructura que se habia construido se desplome y muchos de los
elementos que la conformaban se destruyan en el proceso, aun teniendo un mal
inicio se puede apreciar dos estructuras generadas que logra salvarse, estos dos
tipos de estructuras son interesantes debido a la manera en como estan
conformadas lo cual permite que se mantengan estables a pesar de que la parte
superio de ambas se desplome y pudiera afectar la estabilidad de las mismas. 

El segundo caso mostrado en \ref{fig:exp4_fourth} demuestra el problema del que
se ah estado hablando de los niveles generados pero a una mas grande escala en
donde la mayor parte de los componenetes del nivel aparecen demasiado elevados y
separados entre si ademas de tener el problema del mal acomodo de los mismo lo
cual provoca que terminen cayendo todos al piso y se destruyan en el proceso,
aun asi, como se ah estado tratando de hacer enfasis en estos resultados es que
conjuntos como los que se lograron generar en este nivel no son completamente
malos como los que se pueden ver en las esquinas del nivel arriba de los
elementos de concreto o inclusive el conjunto de la base de la estructura
central, estos elementos pueden ser buenos compuestos para futuras pruebas pero
debido a la manera en como se comportan los niveles en la simulacion son
destruidos y por tal se pierden en futuras generaciones.

Este experimento se puede asemejar mucho al que se discutio en \ref{chap6:exp_2}
en donde los resultados de aptitud de los individuos se mantenia en incrementos
graduales ententualmente logrando llegar a un indiviudo que logra obtener el
mejor promodio de las generaciones, tal es el caso del nivel presentado en
\ref{fig:exp4_fifth} la cual se puede aprecia logra mantener en balance los
componentes del nivel mediate un acomodo que asemeja la apilacion de piezas, a
pesar de tener una distribucion de este tipo se puede observar como algunos
elementos se mantienen en un tipo de balance en sus posiciones debido que
algunas de las bases utilizan elementos demasiado chicos de igual manera que con
estructuras que utilizan piezas triangulares cuando se balancean las demas
estructuras estas tambien se pueden ver afectadas por el cambio de peso hacia
alguno de los lados de los elementos que sostienen y eventualmente caer hacia
los lados, sin embargo debido a que las piezas cuadradas que utilizan en la base
son mas estables que los piezas triangulares entonces es un poco mas complicado
lograr que estos conjuntos se desplomen.

De igual manera se presenta el segundo mejor elemento de la ultima generacion
mostrado en \ref{fig:exp4_sixth}, en este caso el segundo "mejor" elemento de la
ultima generacion esta conformado por estructuras que en su estado inicial estan
mal distribuidas como las que se mostraron anterioremente en
\ref{fig:exp4_third} de igual manera que con los casos anteriores que utilizaban
este tipo de elementos mal acomodados las estructuras terminan cayendo y algunos
componentes terminan destruyendose en el proceso, de igual manera que los
individuos anteriormente presentados se puede aprecia un cierto conjunto de
elementos que logran mantenerse en pie a pesar de los errores de distribucion de
los demas componentes lo cual demuestra al igual que en los ejemplos anteriores
es posible generar conjuntos que pueden ser utilizados en experimentos
subsecuentes para la generacion de niveles.


\import{extras/experimentos/exp_04/}{test.tex}

\newpage

\section{Diversidad del contenido ceado}
\label{section:chap6_diversity_results}

La segunda seccion de resultados se enfoca en los difetenetes resultado
obtenidos de diversidad de los compuestos generados, es decir, se mostraran los
diferentes compuestos que se lograron generar durante las simulaciones del
sistema, el conjunto de ejemplos que se utilizaran para esto se presenta en la
figura \ref{figure:chap6_div_1} en donde se muestran tres grupos de
agrupamientos que se lograron obtener en diferentes experimentos, en estos se
puede mostrar que el sistema es capaz de genrar estructuras que logran ser
interesantes, innovadoras y llamativas para el usuario, de manera independiente
algunas de las estructuras generadas tiene la posibilidad de caer si se les da
el tiempo suficiente para que el desplazamiento de peso logre tomar efecto, sin
embargo muchos de los conjuntos obtenidos pueden ser evolucionados de manera
independiente de tal manera que se pueda obtener elementos como los mostrados en
las figure \ref{figure:chap6_div_2} en los cuales las bases de los compuestos
sean modificadas para que en vez de tener una sola base que puede ser inestables
se tengan dos en bas lados que permitan al compuesto tener la estabilidad
neceasaria para que no se vean tan afectadas por la gravedad del juego o por
acciones del jugaro al estar interactuando en el nivel.

Para estos conjuntos de resultados cabe remarcar que debido a la manera en como
esta codificado el sistema los conjuntos que se presentan como los resultantes
asi como los niveles que terminan conformando son muy dificiles de poder
replicar, es posible replicar partes que conforman las estructuras completas
debido a que muchos de estos son combinaciones a veces sencillas, sin embargo no
se tiene la seguridad de que se logren replicar todas las estructuras en su
totalidad, estos conjuntos obtenidos pueden a su vez ser integrados en listas
extras que permitan que al iniciar experimentos subsecuentes puedan ser
integrados a la lista de compuestos disponibles del experimento, como se explico
anteriomente se proponia que todos los experimentos realizados iniciaran en el
mismo nivel de complejidad es es solamente utilizando los elementos basicos del
juego y cada experimento generara compuestos para utilizar esto permitio que los
niveles que se lograban obtener al final del experimento siempre fuesen
diferentes entre otros experimentos realizados.

\import{extras/experimentos/diversity/}{test.tex}

A pesar de que los resultados que se muestran en esta seccion logran demostrar
que el sistema es capaz tanto de crear niveles diferentes e interesantes asi
como de generar compuestos de estructuras nuevas con la base de las piezas
basicas del juego el sistema aun cuenta con detalles que se pueden pulir para
mejorar el rendimiento del mismo, estos detalles se veran mas detalladamente en
la seccion \ref{chapter:conclusions-and-future-work}.