\chapter{Experimentos y Resultados}
\label{chapter:experiments-and-results}

Este capítulo es el encargado de describir los diferentes experimentos
realizados con el sistema, dichos experimentos se realizaron para comprobar la
correcta funcionalidad de lo creado, así como para revisar que efectivamente el
sistema hace lo que fue propuesto en capítulos anteriores, eso es la generación
de niveles estables y diferentes entre sí.

Estos resultados se presentan mediante el uso de dos
gráficas, la primera establece los valores de fitness mínimos y máximos que se
logran alcanzar durante las generaciones, además de esto en la misma gráfica se
presenta una línea extra que representa el promedio de los fitness de los mismos
individuos, después de esto se presenta una segunda gráfica que representa la
\textit{distancia hamming} mínima alcanzada en cada conjunto de individuos
durante las generaciones, como se explicó anteriormente esta gráfica representa
el valor mínimo de cambios entre los conjuntos de listas de genotipos de los
individuos, esta gráfica tendrá la tendencia de ir reduciéndose conforme avanzan
las generaciones debido a que mientras más se avanza en las generaciones los
elementos elite tenderán a ir apareciendo más.

La manera en la que se define el valor de fitness de los individuos es como se
explica en la sección \ref{subsection:fitness_calculation} es mediante la
separación de los dos aspectos de importancia en los niveles, primero la
\textit{estabilidad} que define el comportamiento del genotipo durante la
simulación y la segunda siendo \textit{diversidad} que define lo \textit{nuevo}
que logran ser los niveles generados.

Los parámetros del algoritmo genético explicados en el capitulo
\ref{chapter:implementation} se muestran en la tabla \ref{table:parametros_ga}.

\begin{table}[ht]
  \caption{Parámetros utilizados en el algoritmo genético}
  \label{table:parametros_ga}
  \centering
  \begin{tabular}{|c|c|}
  \hline
  Parametro & Valor \\
  \hline
  \hline
  Fitness Function & Estabilidad \\ & Diversidad \\
  \hline
  Tamaño de la población & 10 o 20 \\
  \hline
  Numero de Generaciones & 100 o 10 (respectivamente) \\
  \hline
  Criterio de parada & Numero de Generaciones \\
  \hline
  Operador de selección & Selección por torneo \\
  \hline
  Operador de cruce & Cruce de un punto \\
  \hline
  Porcentaje de cruce & 30\% \\
  \hline
  Operador de mutación & Mutación de individuo \\ & Mutación de compuestos \\ & Mutación de material \\
  \hline
  Porcentaje de mutación & 30\% \\
  \hline
  \end{tabular}
\end{table}

El contenido de este capítulo se encargará de mostrar las capacidades de
generación del sistema propuesto e implementado según lo propuesto en el capítulo
\ref{chapter:implementation}.
Cada experimento se evaluará según su capacidad de
generar niveles \textit{estables} así como la capacidad de generar
niveles \textit{diversos}, puesto que se tiene una gran cantidad de individuos
en las simulaciones, entonces para demostrar ambas capacidades se tomarán el
\textit{mejor} y el \textit{peor} del final de cada experimento, así como un
individuo aleatorio de la primera generación. Utilizando estos niveles generados
se explicará la manera en la que evolucionó la diversidad de los niveles y
cuáles fueron los compuestos que se pueden rescatar de las simulaciones, para ser
utilizados en experimentos subsecuentes. Para tener un campo nivelado para todos
los experimentos aquellos compuestos que aparecen durante los experimentos no
son reutilizados en otros experimentos subsecuentes.

Todos los experimentos mostrados en esta sección fueron optimizados durante un
total, de 10 generaciones, esto es debido principalmente a que para realizar las
simulaciones se requiere una gran cantidad de tiempo, esto sumado con la
cantidad diferente de individuos, en experimentos de 100 generaciones se
utilizaban un total de 10 individuos y para experimentos de 10 generaciones se
utilizo un total de 20, además de esto en muchos de los experimentos de 100
generaciones el algoritmo llegaba a un punto de estancamiento generalmente antes
de las primeras 50 generaciones, por tal motivo se decidió reducir la cantidad
de generaciones a 10 pero incrementar el total de individuos para obtener una
mayor diversidad.

Como se explicó en el párrafo anterior los experimentos que se presentarán
cubrirán las variables de 20 individuos durante un total de 10 generaciones y
utilizando las variables presentadas en la tabla \ref{table:parametros_ga}.
Utilizando estas configuraciones se realizaron un total de [Write number here?]
sin embargo, debido a que los resultados mostrados por cada experimento son
demasiados solo algunos resultados de experimentos serán mostrados.

\section{Generación de niveles estables}
\label{section:chap6_stable_level_generation}

La primera sección de resultados se enfoca a mostrar lo que el sistema es capaz
de generar en cuanto a niveles estables se refiere, para este aspecto se tomaron
en cuenta los puntos propuestos en la sección \ref{chapter:proposed-method} en
donde se busca obtener los mejores niveles tomando en cuenta las posiciones de
las piezas colocadas antes y después de las simulaciones siempre teniendo en
cuenta que aquellos niveles que logren mantenerse de pie o inclusive mantener la
mayor cantidad de compuestos aun presentes en el nivel son los que se consideran
como los mejores niveles para cada generación, de la misma manera, cómo se
explico anteriormente aquellos niveles que logren cumplir estas características
son registrados en una lista que mantiene la información de los mejores niveles
generados a lo largo de las generaciones y elementos de este mismo grupo son
reintegrados a la población al iniciar la siguiente generación.

Se representan los resultados de las simulaciones mediante
el uso de 2 gráficas que muestran los valores de la función de aptitud
obtenidos, en esta gráfica se muestra el promedio de los individuos en una
generación dada, el valor mínimo que se alcanza en dicha generación, así como el
valor de aptitud máximo que se logra obtener en las generaciones, en este
aspecto cabe resaltar que obtener no todos los individuos con un valor de 100\%
de aptitud serán los mejores elementos de la población, sino que simplemente la
manera en cómo se generaron los niveles no permite que los compuestos sean
destruidos para esto es necesario revisar de manera visual los resultados y
darse cuenta de que elementos que logran obtener valores diferentes al 100\% de
aptitud pueden ser elementos validos si las estructuras generadas logran cumplir
con los objetivos buscados, después de la gráfica de aptitud de presentan dos
ejemplos de los niveles generados, aquellos denotados por b) representan el
"mejor" individuo de la generación, mientras que aquellos representados por c)
representan el "peor" elemento de la primera generación, estos ejemplos son
representados a su vez de dos maneras, la imagen superior representa el estado
inicial del individuo, es decir al momento de iniciar la simulación, mientras
que la imagen inferior representa el estado final del mismo individuo, es decir
el punto en el que la simulación termino, mediante el uso de estas imágenes es
posible apreciar cómo los niveles generados se comportan en
términos de estabilidad bajo la gravedad proporcionada en el juego.

De igual manera se presenta información de la última generación de la
simulación, en este caso se toman únicamente los dos individuos (diferentes
entre sí) que representan los mejores individuos de la población, en este caso
se toman dos individuos diferentes debido a que debido a la naturaleza de los
algoritmos genéticos la tendencia de los individuos es asemejarse unos a otros y
mediante esto se busca mostrar que es posible inclusive en la generación de
niveles obtener elementos que no necesariamente son los mejores generados peros
logran acercarse lo suficiente a los objetivos del sistema.

\subsection{Experimento 1}
\label{chap6:exp_1}

La Figura \ref{figure:exp_01_a} más específicamente en la parte
\ref{fig:exp1_first} la gráfica de los resultados de aptitud obtenidos durante
las simulaciones, en esta gráfica se puede notar que durante las primeras
generaciones se un elemento de la población siempre mantiene el mismo nivel de
estabilidad, una vista de este elemento de la población se puede aprecia en la
sub Figura \ref{fig:exp1_third} de la misma imagen, así mismo se obtuvo uno de
los "peores" elementos de la generación el cual se muestra en la sub Figura
\ref{fig:exp1_fourth}, como se puede apreciar en la imagen, este elemento de la
población inicia la simulación con varias piezas muy elevadas e inclusive mal
acomodadas, lo cual genera que al terminar la simulación solo una parte de las
estructuras originales se mantenga de pie.

Como es posible apreciar en la gráfica, este conjunto de niveles con dicha
estabilidad se logra mantener durante varias generaciones, no es sino hasta las
últimas dos generaciones, en donde, por medio de las operaciones de cruce y
mutación se obtiene un nuevo individuo capaz de llegar al 80\% de estabilidad.
Este nuevo elemento se puede aprecia en \ref{fig:exp1_fifth}, aquí
mismo se puede aprecia que a pesar de tener el mejor resultado de aptitud de
la generación en ocasiones no es posible mantener el balance en todos los
elementos, sin embargo, muchos se logran mantener sin ser destruidos, de igual
manera se presenta en \ref{fig:exp1_sixth} otro individuo representante de la
ultima generación de la simulación, en este caso este segundo individuo
representa el segundo mejor valor de aptitud que no es completamente igual a
aquel que está en la primera posición, esto es debido a que al integrar miembros
de elite de vuelta a la población muchas veces se integran más de uno que son
individuos completamente iguales debido a que si durante una generación dada el
valor de aptitud no fue mejor que el de un miembro de elite ya existente
entonces el mejor de la generación muchas veces siendo el mismo que se reintegro
a la población es el que se agrega de vuelta a la lista de miembros elite.

\import{extras/experimentos/exp_01/}{test.tex} 
%\subimport{extras/experimentos/exp_01/}{test.tex}

\newpage

\subsection{Experimento 2}
\label{chap6:exp_2}

Como se puede apreciar en la Figura \ref{figure:exp_03_a} o más directamente en
\ref{fig:exp3_first}, este experimento inicio mal, debido a la manera en la que se
generaron los compuestos al iniciar las generaciones, de igual manera este inicio
no tan bueno se puede aprecia en la sub Figura \ref{fig:exp3_third} en donde
varios de los elementos generados no logran mantenerse en buen balance y terminan
cayendo, algunos destruyéndose en el proceso, de igual manera en la sub Figura
\ref{fig:exp3_fourth} del peor individuo de la generación se puede ver que
muchos de los elementos inician su posición en posiciones muy altas o en el caso
mostrado en esta Figura con acomodos muy extraños que terminan por destruir toda
la estructura que se esperaba crear en ese punto.

A pesar de tener mal acomodo y distribución de los elementos en el caso del peor
individuo, se puede apreciar que a pesar de generar estructuras mal ordenadas es
posible generar estructuras que por la manera en la que se acomodan debido a la
gravedad son capaces de mantenerse de pie, este tipo de estructuras aun siendo
visualmente un acomodo "errado" de las piezas puede ser considerado como una
estructura que puede ser rescatada para su uso en futuras generaciones.

A diferencia del primer experimento presentado en esta sección este es capaz de
comenzar a subir el nivel de aptitud a partir de la segunda generación de
individuos, a pesar de ser capaz de subir la aptitud de la primera a la segunda
generación después de eso el sistema no es capaz de continuar mejorando estos
resultados en este caso los resultados se quedan estancados en niveles como los
que se muestran en la sub Figura \ref{fig:exp3_fifth} la cual a pesar de lograr
mantener la mayor cantidad de elementos durante la simulación tiene el problema
de que las piezas inician en posiciones muy elevadas y al momento de ser movidas
por la gravedad pierden su posición original lo cual provoca que el valor de
aptitud se vea afectado, el mejor individuo comparte el mismo problema con el
segundo mejor mostrado en la sub Figura \ref{fig:exp3_sixth} donde se puede
apreciar que las estructuras a pesar de estar aun de pie dado el tiempo
necesario estas estructuras terminaran cayendo hacia un lado u otro, a pesar de
esto este tipo de estructuras generadas permiten ver que existen compuestos en
donde a pesar de mantener los elementos de pie la base de alguna de las
estructuras tiene el comportamiento de una balanza debido al uso de estructuras
triangulares en las cuales un pequeño cambio de peso en alguno de los lados del
mismo provocara que la estructura completa termine cayendo, sin embargo, este
tipo de estructuras proveen un punto de interés debido a lo visualmente
complejo que pueden ser este tipo de estructuras.

\import{extras/experimentos/exp_03/}{test.tex}

\newpage

\subsection{Experimento 3}
\label{chap6:exp_3}

Un tercer grupo de resultados de experimentos se presenta en la Figura
\ref{figure:exp_04_a} más directamente mediante la gráfica de
\ref{fig:exp4_first} este experimento en particular se puede aprecia tuvo un
inicio no muy bueno iniciando en aproximadamente 40\% de aptitud de estabilidad,
esto se puede apreciar más detalladamente en la imagen \ref{fig:exp4_third}
donde se puede ver que el mejor individuo de la primera generación inicia muy
mal la simulación al tener elementos que inicial con muy mal acomodo lo cual
provoca que la estructura que se había construido se desplome y muchos de los
elementos que la conformaban se destruyan en el proceso, aun teniendo un mal
inicio se puede apreciar dos estructuras generadas que logra salvarse, estos dos
tipos de estructuras son interesantes debido a la manera en la que están
conformadas lo cual permite que se mantengan estables a pesar de que la parte
superior de ambas se desplome y pudiera afectar la estabilidad de las mismas. 

El segundo caso mostrado en \ref{fig:exp4_fourth} demuestra el problema del que
se ha estado hablando de los niveles generados, pero a una más grande escala en
donde la mayor parte de los componentes del nivel aparecen demasiado elevados y
separados entre sí además de tener el problema del mal acomodo de los mismos lo
cual provoca que terminen cayendo todos al piso y se destruyan en el proceso.
Aun así, como se ha estado tratando de hacer énfasis en estos resultados es que
conjuntos como los que se lograron generar en este nivel, no son completamente
malos, como los que se pueden ver en las esquinas del nivel arriba de los
elementos de concreto o inclusive el conjunto de la base de la estructura
central. Estos elementos pueden ser buenos compuestos para futuras pruebas pero
debido a cómo se comportan los niveles en la simulación son
destruidos y por lo tanto, se pierden en futuras generaciones. 

Este experimento se puede asemejar mucho al que se discutió en \ref{chap6:exp_2}
en donde los resultados de aptitud de los individuos se mantenían en incrementos
graduales eventualmente logrando llegar a un individuo que logra obtener el
mejor promedio de las generaciones, tal es el caso del nivel presentado en
\ref{fig:exp4_fifth} la cual se puede aprecia logra mantener en balance los
componentes del nivel médiate un acomodo que asemeja la apilacion de piezas, a
pesar de tener una distribución de este tipo se puede observar cómo algunos
elementos se mantienen en un tipo de balance en sus posiciones debido que
algunas de las bases utilizan elementos demasiado chicos de igual manera que con
estructuras que utilizan piezas triangulares cuando se balancean las demás
estructuras estas también se pueden ver afectadas por el cambio de peso hacia
alguno de los lados de los elementos que sostienen y eventualmente caer hacia
los lados, sin embargo, debido a que las piezas cuadradas que utilizan en la base
son más estables que las piezas triangulares entonces es un poco más complicado
lograr que estos conjuntos se desplomen.

De igual manera se presenta el segundo mejor elemento de la última generación
mostrado en \ref{fig:exp4_sixth}, en este caso el segundo "mejor" elemento de la
última generación está conformada por estructuras que en su estado inicial están
mal distribuidas como las que se mostraron anteriormente en
\ref{fig:exp4_third} de igual manera que con los casos anteriores que utilizaban
este tipo de elementos mal acomodados las estructuras terminan cayendo y algunos
componentes terminan destruyéndose en el proceso, de igual manera que los
individuos anteriormente presentados se puede aprecia un cierto conjunto de
elementos que logran mantenerse en pie a pesar de los errores de distribución de
los demás componentes lo cual demuestra al igual que en los ejemplos anteriores
es posible generar conjuntos que pueden ser utilizados en experimentos
subsecuentes para la generación de niveles.

\import{extras/experimentos/exp_04/}{test.tex}

\newpage

\section{Diversidad del contenido creado}
\label{section:chap6_diversity_results}

La segunda sección de resultados se enfoca en mostrar los resultados obtenidos
en cuanto a las creación de diversidad de niveles dentro de los experimentos. La
manera en la que se trata este tipo de generación, es mediante el uso de un valor
de aptitud en los individuos, combinando los puntos establecidos en el capítulo
\ref{chapter:implementation}, específicamente en la sección
\ref{subsection:fitness_calculation}, esto es, los niveles generados toman en
cuenta ambos aspectos de las funciones. Se buscan a los individuos
que logren mantener estabilidad durante las simulaciones y al mismo tiempo que
logren ser diferentes a los niveles que ya se tienen, los puntos que evalúa la función
de fitness son:

\begin{itemize}
  \item Estabilidad
  \item Diferentes compuestos utilizados
  \item Diferencia en la estructura de los niveles
\end{itemize}

El ejemplo mostrado en la Figura \ref{figure:exp_10_a} muestra un experimento en
donde la primera generación crea niveles relativamente diferentes, conforme se
avanza en las generaciones los niveles tienden a asemejarse unos de otros debido
al uso de los miembros elite en las generaciones, sin embargo al final del
experimento se logra obtener un nivel que mantiene un cierto nivel de diferencia
para con los demás de la población. Uno de los puntos que se debe de tratar en
este aspecto es el hecho de que al no tener una facilidad de agregar compuestos
a la lista de posibles compuestos utilizables se tiene el problema de que los
niveles no cambian mucho de manera interna, el método de agregar compuestos
nuevos a la población es mediante el uso de la mutación de los individuos, sin
embargo este solo se realiza con los nuevos individuos y no con aquellos que
siguen existiendo dentro de la población.

\import{extras/experimentos/exp_10/}{test.tex}

El uso de ambas funciones de fitness en la evaluación de los individuos conlleva
un cierto nivel de complejidad, al momento de evolucionar a los individuos. Esto
se debe a que los individuos tratan de ser lo mas diferentes posibles, por medio
de la selección y la combinación, lo cual provoca que los individuos obtenidos en las
ultimas generaciones tengan una tendencia de ser diferentes en cuanto a la
distribución de los compuestos de cada uno. Muchas veces, al tratar de ser
diferentes, generan puntos de inestabilidad en las estructuras generadas. De
igual manera como se explicó anteriormente los individuos obtenidos en estas
evoluciones tienen mayor tendencia a ser similares a aquel con mejor valor de
estabilidad, en este caso la tendencia de este experimento es que los elementos
se mantengan en su mayoría parecidos al mostrado en la parte inferior de la
Figura \ref{fig:exp10_fourth}.

Esta sección se enfoca en los resultados
obtenidos de diversidad de los compuestos generados, es decir, se mostrarán los
diferentes compuestos que se lograron generar durante las simulaciones del
sistema. El conjunto de ejemplos que se utilizarón para esto, se presenta en la
Figura \ref{figure:chap6_div_1} en donde se muestran tres grupos 
que se lograron obtener en diferentes experimentos, en estos se
puede mostrar que el sistema es capaz de generar estructuras que logran ser
interesantes, innovadoras y llamativas para el usuario. De manera independiente,
algunas de las estructuras generadas tienen la posibilidad de caerse, si se les da
el tiempo suficiente. Sin embargo, muchos de los conjuntos obtenidos pueden ser evolucionados de manera
independiente, de tal manera que se pueda obtener elementos como los mostrados en
las Figuras \ref{figure:chap6_div_2} en las que las bases de los compuestos
sean modificadas, para que en lugar de tener una sola base, que la haga inestable,
se tengan dos, en ambos lados. Lo que permite al compuesto tener la estabilidad
necesaria, para que no se vean tan afectadas por la gravedad del juego o por
acciones del jugar, al estar el usuario interactuando con el nivel.

Para estos conjuntos de resultados cabe remarcar que debido a la manera en la que
está codificado el sistema los conjuntos que se presentan como los resultantes
así como los niveles que terminan conformando son muy difíciles de poder
replicar, es posible replicar partes que conforman las estructuras completas
debido a que muchos de estos son combinaciones a veces sencillas, sin embargo, no
se tiene la seguridad de que se logren replicar todas las estructuras en su
totalidad, estos conjuntos obtenidos pueden a su vez ser integrados en listas
extras que permitan que al iniciar experimentos subsecuentes puedan ser
integrados a la lista de compuestos disponibles del experimento, como se explico
anteriormente se proponía que todos los experimentos realizados iniciaran en el
mismo nivel de complejidad es solamente utilizando los elementos básicos del
juego y cada experimento generara compuestos para utilizar esto permitió que los
niveles que se lograban obtener al final del experimento siempre fuesen
diferentes entre otros experimentos realizados.

\import{extras/experimentos/diversity/}{test.tex}

%% NO ESTA CLARO
A pesar de que los resultados que se muestran en esta sección logran demostrar
que el sistema es capaz tanto de crear niveles diferentes e interesantes así
como de generar compuestos de estructuras nuevas con la base de las piezas
básicas del juego el sistema aun cuenta con detalles que se pueden pulir para
mejorar el rendimiento del mismo, estos detalles se verán más detalladamente en
la sección \ref{chapter:conclusions-and-future-work}.
