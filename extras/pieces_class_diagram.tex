%%%%%%%%%%%%%%%%%%%%%%%%%%%%%%%%%%%%%%%%%%%%%%%%%%%%%%%%%%%%%%%
% Class diagram
% Author: Salinas Hernández Jaime
% Version: 1.0 
%%%%%%%%%%%%%%%%%%%%%%%%%%%%%%%%%%%%%%%%%%%%%%%%%%%%%%%%%%%%%%%
\tikzstyle{abstract}=[rectangle, draw=black, rounded corners, fill=blue!40, drop shadow,
        text centered, anchor=north, text=white, text width=4.2cm]
\tikzstyle{comment}=[rectangle, draw=black, rounded corners, fill=green, drop shadow,
        text centered, anchor=north, text=white, text width=4.2cm]
\tikzstyle{myarrow}=[->, >=open triangle 90, thick]
\tikzstyle{line}=[-, thick]
        

\begin{tikzpicture}[node distance=2cm]
    \node (Item) [abstract, rectangle split, align=left, rectangle split parts=3]
        {
            \textbf{Pieza}
            \nodepart{second}string: Material \newline float: X \newline float: Y \newline float: Z \newline
            \nodepart{third}get\_edges \newline as\_dictionary \newline get\_points \newline update\_values
        };
    %\node (ItemInstants) [comment, rectangle split, rectangle split parts=2, below=0.2cm of Item, text justified]
    %    {
    %        \textbf{Methods}
    %        \nodepart{second}
    %            get\_edges
    %            \newline as\_dictionary
    %            \newline get\_points
    %            \newline update\_values
    %    };
    \node (AuxNode01) [text width=4cm, below=3cm of Item] {};
    
    \node (Circle) [abstract, rectangle split, align=left, rectangle split parts=2, left=of AuxNode01]
        {
            \textbf{Circle}
            \nodepart{second}text: "Circle" \newline
            int: Height = 75 \newline
            int: Width = 75
        };
    \node (RectTiny) [abstract, rectangle split, align=left, rectangle split parts=2, right=of Circle]
        {
            \textbf{RectTiny}
            \nodepart{second}text: "RectTiny"\newline int: Height = 25 \newline int: Width = 45
        };
    \node (RectSmall) [abstract, rectangle split, align=left, rectangle split parts=2, right=of RectTiny]
        {
            \textbf{RectSmall}
            \nodepart{second}text: "RectSmall"\newline int: Height = 25 \newline int: Width = 85
        };
    \node (RectMedium) [abstract, rectangle split, align=left, rectangle split parts=2, right=of RectSmall]
        {
            \textbf{RectMedium}
            \nodepart{second}text: "RectMedium"\newline int: Height = 25 \newline int: Width = 165
        };
    \node (RectBig) [abstract, rectangle split, align=left, rectangle split parts=2, right=of RectMedium]
        {
            \textbf{RectBig}
            \nodepart{second}text: "RectBig"\newline int: Height = 25 \newline int: Width = 185
        };
    
    
        
    \node (AuxNode02) [text width=0.5cm, below=of Circle] {};   
    
    \node (RectFat) [abstract, rectangle split, align=left, rectangle split parts=2, left=of AuxNode02]
        {
            \textbf{RectFat}
            \nodepart{second}text: "RectFat"\newline int: Height = 25 \newline int: Width = 85
        };
    \node (SquareTiny) [abstract, rectangle split, align=left, rectangle split parts=2, right=of RectFat]
        {
            \textbf{SquareTiny}
            \nodepart{second}text: "SquareTiny"\newline int: Height = 25 \newline int: Width = 25
        };
    \node (SquareSmall) [abstract, rectangle split, align=left, rectangle split parts=2, right=of SquareTiny]
        {
            \textbf{SquareSmall}
            \nodepart{second}text: "SquareSmall"\newline int: Height = 45 \newline int: Width = 45
        };
    \node (Triangle) [abstract, rectangle split, align=left, rectangle split parts=2, right=of SquareSmall]
        {
            \textbf{Triangle}
            \nodepart{second}text: "Triangle"\newline int: Height = 75 \newline int: Width = 75
        };
    \node (TriangleHole) [abstract, rectangle split, align=left, rectangle split parts=2, right=of Triangle]
        {
            \textbf{TriangleHole}
            \nodepart{second}text: "TriangleHole"\newline int: Height = 85 \newline int: Width = 85
        };
    \node (SquareHole) [abstract, rectangle split, align=left, rectangle split parts=2, right=of TriangleHole]
        {
            \textbf{SquareHole}
            \nodepart{second}text: "SquareHole"\newline int: Height = 85 \newline int: Width = 85
        };
        
    
    
    \draw[myarrow] (RectSmall.north) -- ++(0,0.8) -| (Item.south);
    \draw[line] (Circle.north) -- ++(0,0.8) -| (RectTiny.north);
    \draw[line] (Circle.north) -- ++(0,0.8) -| (RectSmall.north);
    \draw[line] (Circle.north) -- ++(0,0.8) -| (RectMedium.north);
    \draw[line] (Circle.north) -- ++(0,0.8) -| (RectBig.north);
    \draw[line] (Circle.north) -- ++(0,0.8) -| (RectFat.north);
    \draw[line] (Circle.north) -- ++(0,0.8) -| (SquareTiny.east);
    \draw[line] (Circle.north) -- ++(0,0.8) -| (SquareSmall.east);
    \draw[line] (Circle.north) -- ++(0,0.8) -| (Triangle.east);
    \draw[line] (Circle.north) -- ++(0,0.8) -| (TriangleHole.east);
    \draw[line] (Circle.north) -- ++(0,0.8) -| (SquareHole.east);
        
        
\end{tikzpicture}