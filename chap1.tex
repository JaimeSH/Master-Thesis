\chapter{Introduction}
\label{chapter:introduction}

El tema de generación de contenido procedural es un tema que ha sido tomado con
gran interés en el ámbito de desarrollo de videojuegos debido a las facilidades
que brinda a las compañías para poder generar contenido que puede ser utilizado
en juegos, no es posible ligar la generación procedural de contenido con algún
aspecto especifico de la creación de videojuegos o inclusive únicamente en el
enfoque de videojuegos debido a que el tema puede ser utilizado para varios
aspectos diferentes, desde la generación de áreas como en el videojuego Spelunky
\cite{RovioEntertainmentCorporation2009} \cite{Mossmouth2013}
\cite{Mossmouth2013}, la generación de historias que sean inmersivas como en
Façade \cite{Mateas}, la generación de piezas sonoras como las que son posibles
de escuchar en Audioverdrive \cite{Holtar}, generación de efectos gráficos como
el diseño de pétalos en Petalz \cite{Risi2012}, así como otros en los enfoques
de gameplay y videojuegos completos.

Estas ideas han sido de gran interés mutuo entre la industria y la comunidad
científica que en 2005 se originó un simposio de la IEEE en el ámbito de
Computación Inteligente y Videojuegos (Computational Intelligence and Games) el
cual posteriormente pasaría a ser una conferencia en el año 2009.

Esta conferencia tiene el fin de unir a ambos grupos de interés en un mismo
lugar con el fin de conocer sobre los avances de la computación inteligente
enfocada en juegos, dentro del marco de la misma conferencia se celebran
competencias de diferentes juegos tales como Inteligencia artificial en el juego
de Hearthstone, jugar niveles de Ms Pac-Man, competencias de agentes para ganar
en juegos de pelea y varios otros.

Dentro de estas mismas competencias se celebra la de generación de niveles para
el videojuego de Angry Birds la cual consiste primordialmente en generar niveles
que sean interesantes visualmente y sean complejos en el sentido de que sean
difíciles de completar por una persona pero que al mismo tiempo no sean
imposibles de completar, este juego fue desarrollado por la compañía Rovio
Entertainment Corporation\cite{RovioEntertainmentCorporation2009} en el año
2009, el objetivo dentro del juego consta de utilizar un numero dado de aves con
diferentes efectos para eliminar puercos color verde que se encuentran
protegidos eh inclusive sobre estructuras de diferentes materiales y formas, en
algunos casos al golpear partes de las estructuras se crea un efecto domino que
termina por destruir la mayoría de las misma, el juego en cuestión cuenta con un
sistema de gravedad que permite que las trayectorias de tiro generen diferentes
efectos en las estructuras, en este caso el interés principal es que mediante el
uso del sistema de gravedad del mismo juego se puedan crear estructuras
interesantes visualmente y que sean lo suficientemente robustas para soportar
golpes de las mismas aves y logre mantenerse de pie lo más posible.

El objetivo principal de este proyecto es el de generar un sistema basado en
computo evolutivo que logre generar las estructuras que conformaran los niveles
del juego de Angry Birds y evaluar estas mismas estructuras mediante los
aspectos propuestos en la competencia para que sean entretenidos y complejos con
un cierto grado de dificultad. De una manera más estructurada los puntos
objetivo que se enmarcaran en el proyecto son los siguientes: 

\begin{itemize}
  \item Adaptar un algoritmo genético para que sea capaz de generar secuencias
  de estructuras para un nivel mediante objetivos propuestos.
  \item Modificar y adaptar el software de simulación para poder obtener datos
  específicos de un nivel para poder evaluarlo.
  \item Adaptar el método de evolución para permitir que se puedan generar
  estructuras complejas.
  \item Explorar las posibilidades de integrar alguna otra técnica de computo
  evolution para optimizar la generación de niveles.
\end{itemize} 

El trabajo propuesto en este trabajo de tesis tiene como objetivo la generacion
de el contenido con el cual un jugador podra interactuar en el videojuego de
angry birds, cabe remarcar que el uso de metodos de generacion procedural han
sido utilizados con el proposito de realizar competencias en el marco de la
conferencia del Instituto de Ingeniería Eléctrica y Electrónica (IEEE) que lleva
por nombre Conferencia de Juegos (CoG) en dicha competencia se utilizan
algoritmos capaces de generar los niveles que una persona o agente puede jugar.
En nuestro caso la metodologia propuesta en este proyecto pretende servir de
base para mostrar que es posible la utilizacion de algoritmos evolutivos libres
de tal manera que el contenido generado sera unico y entretenido para los
usuarios. En este mismo capitulo se presentan las motivaciones para la
realizacion del proyecto asi como la manera en que la computacion evolutiva
sirve de apoyo en la industria del entretenimiento (ver la Seccion
\ref{section:justification}).

El metodo propuesto en este documento consta de tres campos diferentes, estos
campos son Generacion de contenido procedural y dos areas de la computacion
evolutiva las cuales son algoritmos geneticos y open-ended evolution. El
Capitulo \ref{chapter:preliminaries} esta dedicado para explicar los conceptos
de estas areas, mismos que son necesarios para comprender mejor el contenido de
los capitulos subsecuentes a ese asi como de todo el proyecto de tesis.

Una vez que los conceptos mostrados en el capitulo anterior han sido
comprendidos el Capitulo \ref{chapter:related-work} procede a presentar un
conjuntos de trabajos relacionados con lo que se quiere lograr. Estos trabajos
permiten tener un mejor entendimiento de que es lo que se puede llegar a lograr
con las areas presentadas asi como el comprender las diferentes maneras que
existen para trabajar en el proyecto.

Posteriormente en el Capitulo \ref{chapter:proposed-method} se presenta la
propuesta del proyecto de tesis, en este capitulo se muestra el como nuestra
propuesta sirve para resolver el problema de generacion de contenido para el
juego. De igual manera se toca el tema de las propuestas que previamente se
habian contemplado para resolver el mismo problea y como fue que mediante la
modificacion y adaptacion de conceptos de estas ideas se logro llegar a la
propuesta definitiva.

La implementacion particular del metodo propuesto se presenta en el Capitulo
\ref{chapter:implementation}, este capitulo cubre todo el aspecto tecnico del
proyecto tal como los lenguajes de programacion utilizados, las adapataciones
que realizaron a los archivos de ejecucion del software que se utiliza para la
simulacion asi como la manera en el como las ideas propuestas en el Capitulo
\ref{chapter:proposed-method} fueron implementadas en el codigo, debido a que
los conceptos que se presentaron en el Capitulo \ref{chapter:proposed-method}
son adaptadas en este, varios nombres de secciones tenderan a ser similares.

Despues de realizar las implementaciones pertinentes de las ideas presentadas en
el Capitulo \ref{chapter:proposed-method} se procede a realizar un conjunto de
experimentos del sistema funcional. Los experimentos permiten comprender la
menera en como se puede realizar la generacion de contenido con los metodos
propuestos ademas de que los resultados obtenidos sirven de apoyo para analizar
las partes en donde se pueden realizar mejorar para un mejor desempeño. Estos
experimentos se presentan en el Capitulo \ref{chapter:experiments-and-results},
en este mismo capitulo se analizan los resultados y se dan comparaciones con
diferentes variaciones del metodo y con otras idead de los trabajos presentados
en el Capitulo \ref{chapter:related-work}.

Finalmente en el Capitulo \ref{chapter:conclusions-and-future-work} se presentan
la conclusiones a las que se logro llegar despues de concluir el trabajo asi
como varias propuestas que se pueden agregar o secciones que se puecen optimizar
para mejorar el proyecto en un futuro.

Este proyecto de tesis describe un metodo innovador para la generacion de
estructuras del juego mediante el uso de tecnicas evolutivas como la convinacion
de un algoritmo genetico con un algoritmo de evolucion libre. Este metodo es
robusto y permite modificaciones rapidas a las caracteristicas evolutivas y de
evaluacion, ademas de que es flexible parpoder establecer restricciones en el
contenido generado. Por ejemplo que no se utilizen ciertas piezas o que se
generen niveles con cierta cantidad de objetivos.

\section{Justificacion}
\label{section:justification}

La justificación de este proyecto se enfoca en dos puntos diferentes, el primero
tomando en cuenta el punto de vista científico por el cual se tiene interés
particular en el proyecto en el cual la utilización de generación procedural de
contenido en este caso de generación de niveles es un caso de estudio muy
particular debido a las restricciones que se deben de tomar en cuenta al momento
del diseño debido a que varios aspectos tales como los diferentes terrenos, el
posicionamiento y balance de los elementos colocados deben de ser tomados en
cuenta de una manera que se asemeje a una manera que podrían ser posicionados en
el mundo real con restricciones de gravedad para poder generar estructuras que
logren cumplir los aspectos denotados por la jugabilidad que hacen a un nivel
viable e interesante, mientras que el segundo aspecto de la justificación se
enfoca al ámbito tecnológico que toma el proyecto para las empresas que utilizan
este tipo de recursos en el diseño de videojuegos, esto es debido a que las
mismas empresas muchas veces se ven presionados en cumplir fechas de salida para
dichos productos de esta manera el uso de este tipo de recursos permite generar
el contenido de manera automatizada y que logre cumplir con los estándares
establecidos por las compañías.

El problema presentado al inicio de este Capitulo se puede dividir dos partes:
un algoritmo de generacion de compuestos que requiere considerar 1) que los
conjuntos generados logren mantenerse estables, 2) que los conjuntos generados
no sean igual a otros previamente generados y 3) que estos conjuntos puedan
hacer uso de los ya existentes y puedan alimentar en paralelo una lista con los
conjuntos que pueden ser utilizados por el algoritmo principal, mientras que la
segunda parte la cual es el sistema evolutivo debe de poder utilizar los
compuestos obtenidos por la part anterior para poder generar estructuras que
puedan mantenerse de pie y aumentar su complejidad mediante las estructuras.

Se propone, mediante el uso de un algoritmo genetico agregarle complejidad a un
nivel mediante la evolucion de las estructuras que lo componen, esto debido a
que es posible realizar operaciones de cruce y mutaciones que lograran producir
nuevas estructuras como posibles soluciones ademas de que permite encaminar el
sistema a encontrar las mejores mediante funciones objetivo. Esto es interesante
debido a las posibilidades de generacion del sistema debido a que es posible
obtener tanto estructuras como combinaciones de elementos que normalmente una
persona no utilizaria de manera manual.

Finalmente, el metodo propuesto provee un sistema que es capaz de crear niveles
que seran entretenidos, unicos eh interesantes para los jugadores.
 