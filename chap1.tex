\chapter{Introduction}
\label{chapter:introduction}

El método de generación de contenido procedural ha sido tomado con
gran interés en el ámbito de desarrollo de videojuegos debido a los beneficios
que brinda a las compañías para poder generar contenido que pueda ser utilizado
en juegos. No es posible asociar la generación procedural de contenido con algún
aspecto especifico de la creación de videojuegos o inclusive únicamente en el
enfoque de videojuegos debido a que el tema puede ser utilizado para 
aspectos diferentes, desde la generación de áreas como en el videojuego Spelunky
\cite{RovioEntertainmentCorporation2009} \cite{Mossmouth2013}
\cite{Mossmouth2013}, la generación de historias, que sean inmersivas como en
Façade \cite{Mateas}, la generación de piezas sonoras como las que se pueden 
escuchar en Audioverdrive \cite{Holtar}, generación de efectos gráficos como
el diseño de pétalos en Petalz \cite{Risi2012}, así como otros en los enfoques
de gameplay y videojuegos completos.

Estas ideas han sido de mutuo interés entre la industria y la comunidad
científica por lo que en 2005 se originó un simposio de la IEEE en el ámbito de
Computación Inteligente y Videojuegos (Computational Intelligence and Games) el
cual posteriormente pasaría a ser una conferencia en el año 2009.

Esta conferencia tiene el fin de unir a ambos grupos de interés en un mismo
lugar con el fin de conocer sobre los avances de la computación inteligente
enfocada en juegos, dentro del marco de la misma conferencia se celebran
competencias de diferentes juegos tales como inteligencia artificial en el juego
de Hearthstone, jugar niveles de Ms. Pac-Man, competencias de agentes para ganar
en juegos de pelea y varios otros.

Dentro de estas mismas competencias se celebra la de generación de niveles para
el videojuego de Angry Birds la cual consiste primordialmente en generar niveles
que sean interesantes visualmente y sean complejos en el sentido de que sean
difíciles de completar por una persona pero que al mismo tiempo no sean
imposibles de completar, este juego fue desarrollado por la compañía Rovio
Entertainment Corporation\cite{RovioEntertainmentCorporation2009} en el año
2009, el objetivo dentro del juego consta de utilizar un número dado de aves con
diferentes efectos para eliminar puercos color verde que se encuentran
protegidos e incluso sobre estructuras de diferentes materiales y formas, en
algunos casos al golpear partes de las estructuras se crea un efecto dominó que
termina por destruir la mayoría de las misma, el juego cuenta con un
sistema de gravedad que permite que las trayectorias de tiro generen diferentes
efectos en las estructuras. En este trabajo el interés principal es que mediante el
uso del sistema de gravedad del juego se puedan crear estructuras
interesantes visualmente y que sean lo suficientemente robustas para soportar
golpes de las mismas aves y logre mantenerse en pie el mayor tiempo posible.

El objetivo principal de este proyecto es el de generar un sistema basado en
computo evolutivo que logre generar las estructuras que conformarán los niveles
del juego de Angry Birds y evaluar estas mismas estructuras mediante los
aspectos propuestos en la competencia para que sean entretenidos, complejos y con 
cierto grado de dificultad.  Los objetivos que se enmarcaran en el proyecto son los siguientes: 

\begin{itemize}
  \item Adaptar un algoritmo genético para que sea capaz de generar secuencias
  de estructuras para generar un nivel cumpliendo los requerimientos.
  \item Modificar y adaptar el software de simulación para poder obtener datos
  específicos de un nivel requeridos para su evaluación.
  \item Adaptar el método de evolución para permitir que se puedan generar
  estructuras complejas.
  \item Explorar las posibilidades de integrar alguna otra técnica de computo
  evolutivo para optimizar la generación de niveles.
\end{itemize} 

El trabajo propuesto tiene como objetivo la generación
de contenido con el cual un jugador podrá interactuar en el videojuego de
Angry Birds, cabe remarcar que el uso de métodos de generación procedural han
sido utilizados con el proposito de realizar competencias en el marco de la
conferencia del Instituto de Ingeniería Eléctrica y Electrónica (IEEE) que lleva
por nombre Conferencia de Juegos (CoG), en dicha competencia se utilizan
algoritmos capaces de generar los niveles que una persona o agente puede jugar.
En nuestro caso la metodología propuesta en este proyecto pretende servir de
base para mostrar que es posible la utilizacion de algoritmos evolutivos libres
de tal manera que el contenido generado será único y entretenido para los
usuarios. En este mismo capítulo se presentan las motivaciones para la
realización del proyecto así como la manera en que la computación evolutiva
sirve de apoyo en la industria del entretenimiento (ver la Sección
\ref{section:justification}).

El método propuesto en este documento consta de tres campos diferentes, estos
campos son Generación de Contenido Procedural y dos áreas de la computación
evolutiva las cuales son algoritmos genéticos y la evolución abierta. El
Capítulo \ref{chapter:preliminaries} está dedicado a explicar los conceptos
de estas áreas, mismos que son necesarios para comprender mejor el contenido de
los capítulos subsecuentes.

Una vez que los conceptos mostrados en el capítulo anterior han sido
vistos en el Capítulo \ref{chapter:related-work} se presentan un
conjuntos de trabajos relacionados con lo que se quiere lograr. Estos trabajos
permiten tener un mejor entendimiento de que es lo que se puede llegar a lograr
con los métodos presentados así como el comprender las técnicas 
existentes para trabajar en el proyecto.

Posteriormente en el Capítulo \ref{chapter:proposed-method} se presenta la
propuesta del proyecto de tesis, y el enfoque elegido para resolver el 
problema de generación de contenido para el
videojuego de Angry Birds. De igual manera se toca el tema de las 
propuestas que previamente se habian presentado para resolver el mismo 
problema y como fue que mediante la modificación y adaptación de conceptos
de estas ideas se logró llegar a la propuesta definitiva.

La implementacion particular del método propuesto se presenta en el Capítulo
\ref{chapter:implementation}, este capitulo cubre todo el aspecto técnico del
proyecto tal como los lenguajes de programación utilizados, las adapataciones
que realizaron a los archivos de ejecución del software que se utiliza para la
simulación así como la manera en la que las ideas propuestas en el Capítulo
\ref{chapter:proposed-method} fueron implementadas en el código, debido a que
los conceptos que se presentaron en el Capítulo \ref{chapter:proposed-method}
son adaptadas en este.

Después de realizar las implementaciones pertinentes de las ideas presentadas en
el Capítulo \ref{chapter:proposed-method} se procede a realizar un conjunto de
experimentos del sistema funcional. Los experimentos permiten comprender la
menera de realizar la generacion de contenido con los métodos
propuestos, además de que los resultados obtenidos sirven de apoyo para analizar
las partes en dónde se pueden realizar mejoras para un mejor desempeño. Estos
experimentos se presentan en el Capítulo \ref{chapter:experiments-and-results},
en este mismo capítulo se analizan los resultados y se dan comparaciones con
diferentes variaciones del método y con otras ideas de los trabajos presentados
en el Capítulo \ref{chapter:related-work}.

Finalmente en el Capítulo \ref{chapter:conclusions-and-future-work} se presentan
la conclusiones a las que se logró llegar después de concluir el trabajo así
como varias propuestas que se pueden agregar o secciones que se pueden optimizar
para mejorar el proyecto en un futuro.

Este proyecto de tesis describe un método innovador para la generación de
estructuras del juego mediante el uso de técnicas evolutivas como la convinación
de un algoritmo genético con un algoritmo de evolución abierta. Este método es
robusto y permite modificaciones rápidas a las características evolutivas y de
evaluación, además de que es flexible para poder establecer restricciones en el
contenido generado. Por ejemplo que no se utilizen ciertas piezas o que se
generen niveles con cierta cantidad de objetivos.

\section{Justificación}
\label{section:justification}

La justificación de este proyecto se presenta en dos puntos diferentes. 

\begin{itemize}
  \item Tomando en cuenta el punto de vista científico, por el cual
  se tiene interés particular en el proyecto. La utilización de métodos de
  generación procedural de contenido en este caso de generación de niveles es un
  caso de estudio muy particular debido a las restricciones que se deben de
  tomar en cuenta al momento del diseño, debido a que varios aspectos tales como
  los diferentes terrenos, el posicionamiento y balance de los elementos
  colocados deben de ser tomados en cuenta de una manera que se asemeje a una
  manera que podrían ser posicionados en el mundo real con restricciones de
  gravedad para poder generar estructuras que logren cumplir los aspectos
  denotados por la jugabilidad que hacen a un nivel viable e interesante.
  \item Ambito tecnológico que toma el proyecto para las empresas que utilizan
  este tipo de recursos en el diseño de videojuegos, esto es debido a que las
  mismas empresas muchas veces se ven presionados en cumplir fechas de salida para
  dichos productos de esta manera el uso de este tipo de recursos permite generar
  el contenido de manera automatizada y que logre cumplir con los estándares
  establecidos por las compañías.
\end{itemize}
 
% Te recomiendo que este sea otro punto, pero utilizando viñetas. 
% También no queda claro el párrafo que sigue.
%mientras que el segundo aspecto de la justificación se
%enfoca al 

% Tal vez este párrafo va en otro lado, igual muy al principio cuando explicas 
% la propuesta.

%El problema presentado al inicio de este capitulo se puede dividir dos partes:
%un algoritmo de generación de compuestos que requieren considerar 1) que los
%conjuntos generados logren mantenerse estables, 2) que los conjuntos generados
%no sean igual a otros previamente generados y 3) que estos conjuntos puedan
%hacer uso de los ya existentes y puedan alimentar en paralelo una lista con los
%conjuntos que pueden ser utilizados por el algoritmo principal, mientras que la
%segunda parte la cual es el sistema evolutivo debe de poder utilizar los
%compuestos obtenidos por la part anterior para poder generar estructuras que
%puedan mantenerse de pie y aumentar su complejidad mediante las estructuras.

Se propone, mediante el uso de un algoritmo genetico agregarle complejidad a un
nivel mediante la evolucion de las estructuras que lo componen, esto debido a
que es posible realizar operaciones de cruce y mutaciones que lograran producir
nuevas estructuras como posibles soluciones ademas de que permite encaminar el
sistema a encontrar las mejores mediante funciones objetivo. Esto es interesante
debido a las posibilidades de generacion del sistema debido a que es posible
obtener tanto estructuras como combinaciones de elementos que normalmente una
persona no utilizaria de manera manual.

Finalmente, el metodo propuesto provee un sistema que es capaz de crear niveles
que seran entretenidos, unicos eh interesantes para los jugadores.

En los capítulos siguientes se presentan las bases teóricas que se tomaron en
cuenta para el desarrollo de este proyecto, los diferentes métodos que se han
utilizado para atacar la problemática descrita, así como la propuesta que
definimos para resolver este mismo problema junto son sus resultados y
conclusiones.