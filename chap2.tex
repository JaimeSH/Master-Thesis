\chapter{Preliminaries}
\label{chapter:preliminaries}

En este capítulo se explican los conceptos básicos de los temas que toman un
papel importante en el desarrollo del proyecto separados cada uno en secciones,
principalmente algoritmos genéticos siendo que este tema es la parte primordial
del proyecto, además de esto se explica de manera detallada los conceptos que se
tomaran en cuenta en el desarrollo de los métodos que componen el sistema,
siendo estos open-ended evolution, generación procedural de contenido, los
conceptos básicos de jugabilidad.

\section{Algoritmos Geneticos}
\label{section:genetic-algorithms}

Los algoritmos geneticos (Genetic Algorithm, GA por sus siglas en ingles) son
algoritmos inspirados en la evolucion Darwiniana y utilizados para la
optimizacion de procesos, fue introducido por John Holland en 1975
\cite{Holland1975}, los GA son metodos de optimizacion y busqueda basados en los
principios de la seleccion natural y genetica, la manera de representarlos es
mediante un grupo de individuos que representan posibles soluciones a un
problema y mediante el uso de operadores geneticos estos individuos se cruzan y
evolucionan para acercarse mas al objetivo, el cual puede ser minimizar o
maximizar un valor resultado de un sistema.

Deacuerdo a una explicacion proporcionada por Whitley en uno de sus articulos
\cite{Whitley1994}, un algoritmo genetico (GA) son algoritmos de optimizacion
inspirados en el proceso natural de evolucion. Por esta razon los GAs son
considerados como parate de una sub-area de algoritmos de optimizacion llamada
algoritmos evolutivos. La manera en como funciona un GA es que se propone tener
un conjunto inicial de posibles soluciones a problema, estas soluciones son
llamadas poblacion dentro del algoritmo. Cada una de estas soluciones propuestas
lleva el nombre de individuos y en ocaciones se definen como cromosomas. Esta
poblacion es evaluada mediante el uso de una funcion de aptitud encargada de
determinar el rendimiento de cada uno de los individuos como una de las posibles
soluciones para el problema que se esta tratando. Este tipo de problemas
comunmente involucran encontrar un conjunto de parametros que permitan minimizar
o maximizar los resultados obtenidos en un sistema.

El proceso que se realiza en un GA se describe de la siguiente manera, primero
la poblacion es sometida a un proceso de evolucion que involucra la realizacion
de cruces entre los individuos de la poblacion que tienen el mejor rendimiento
en la generacion, de igual manera aquellas soluciones que tengan un mal
desempeño deacuerdo a la funcion de aptitud son removidos de la lista de la
poblacion para las generaciones subsecuentes. El proceso de cruce de individuos
se realiza mediante la recombinacion de geners en los cromosomas de los
individuos seleccionados los cuales son seleccionados en base a sus resultados
en la funcion de aptitud siendo aquellos que llevaran a cabo los cruces los
mejores. Este proceso mencionado se repite durante un numero determinado de
iteraciones las cuales llevan el nombre de generaciones.

El objetivo de los GAs asi como en otros algoritmos evolutivos es que, mientras
las generaciones van pasando la poblacion comenzara a volverse mas apta y se
encaminara a cierto punto o valor que se espera sea el indicado para resolver un
problema, sin embargo, es comun que ninguno de los individuos logre llegar a una
solucion optima para el problema presentado. Sin embargo, al ejecutar el
algoritmo multiples veces el algoritmo evolutivo lograra proveer varios
resultados diferentes, esto es debido a que la poblacion inicial en cada ciclo o
ejecucion se genera de manera aleatoria. Por esta razon, los algoritmos
evolutivos y por ende los algoritmos geneticos son considerados como algoritmos
de busquedas meta-heuristicas, debido a que son capaces de encontrar soluciones
casi optimas para problemas en donde los algoritmos evolutivos no sean
involucrados de manera directa, ademas de esto los algoritmos evolutivos tambien
son considerados algoritmos estocasticos debido a la aleatoriedad que conlleva
utilizar este tipo de algoritmos \cite{Harik1999}.

De esta manea el ciclo de vida de un algoritmo genetico se resume de la
siguiente manera: 

\begin{enumerate}
    \item Inicializar una poblacion aleatoria de n individuos.
    \item Evaluar la aptitud de cada individuo(solucion) en la poblacion.
    \item Revisar si se ah llegado a una condicion de terminacion(el valor buscado)
    \item En caso de que no
    \begin{enumerate}
        \item Introducir nuevos individuos a la poblacion mediante los
        operadores de seleccion y cruce.
        \item Mutar de manera alatoria a uno de los individuos nuevos.
    \end{enumerate} 
    \item Repetir desde el punto 2 hasta llegar a la solucion o se cumpla el
    limite de generaciones.
    \item Seleccionar el mejor individuo de la poblacion como la solucion del problema.
\end{enumerate}

\begin{figure}
    \centering
    \includegraphics[width=0.8\textwidth]{img/ga_life_cycle.png}
    \caption{Ciclo de vida de un algoritmo genetico}
    \label{figure:GA-Cycle}
\end{figure}

Esta lista de pasos se muetra graficamente en la imagen \ref{figure:GA-Cycle} en
donde se puede apreciar mas claramente como los operadores de seleccion, cruce y
mutacion se engloban en una sola area que es la de la generacion e integracion
de nuevos individuos, estos mismos nuevos individuos tambien se evaluan para
definir si el algorito ah logrado alcanzar el punto de terminacion buscado, como
se explico anteriormente estos nuevos individuos no se eliminan en caso de no
haber sido los ideales sino que se comparan con los individuos originales y en
caso de ser mejores iran recorriendo a los peores con el fin de encaminar al
algoritmo a un punto especifico que se espera sea la solucion idonea.

\section{Generación procedural de contenido}
\label{section:PCG}

El término "Generacion procedural de contenido" (Procedural Content Generation o
PCG por sus siglas en ingles) denota la manera de crear contenido de manera
automatica mediante algoritmos en lugar de generar los mismos contenidos de
manera manual.

La PCG es un sistema de generacion de contenido utilizado desde finales de los
70s, inicialmente se utilizaba para generar los laberintos de algunos juegos
utilizando arte ASCII donde se denotaba la distribucion de los cuartos y objetos
que se podian encontrar en juegos estilo \textit{"rogue"} o juegos que simulan
un juego RPG de mesa, actualmente es ambito del la generacion de contenido ah
tomado varias variantes y cada vez mas empresas toman un interes por el uso de
este tipo de herramientas de diseño, cabe remarcar que los diferentes ambitos no
estan del todo refinados sin embargo dada la informacion necesaria y aplicados a
aspectos especificos en el ciclo de desarrollo son capaces de crear los
contenidos que se requieran.

Mientras que la generacion de contenido es un aspecto que se ah utilizado en
muchos ambitos diferentes como el fotografia, video, anuncios y arte digital, en
el area de videojuegos se maneja el uso de generacion "procedural" de contenido
en donde procedural se define como el proceso computacional de una funcion
particular, en este caso lo que se busca con la generacion procedural de
contenido es reducir el tiempo que toma generar contenidos de manera manual, la
siguiente seccion explica mas detalladamante como se utiliza esta mecanica en el
area de videojuegos.

\subsection{PCG en el ámbito de videojuegos}
\label{subsection:PCGInGames}

Dentro del ámbito de videojuegos la generación de contenido procedural es un
aspecto que ha tenido un gran impulso en tiempos recientes debido a que muchas
empresas de preocupan por sacar al mercado juegos de manera continua, estos
mismos muchas veces en ciclos de desarrollos muy cortos, por tal se ha buscado
diferentes maneras de reducir dichos problemas, una herramienta útil que ha
surgido debido a esto es la generación procedural la cual permite reducir no
solo los tiempos de desarrollo sino que también permite recudir el espacio de
memoria total de un juego en particular debido a que ya no es necesario tener
todos los archivos englobados en un solo lugar sino que lo requerido se obtiene
de manera automática, de esta misma manera se le permite a las empresas reducir
el número de personal necesario y por tal reducir los costos de desarrollo.

De una manera simple la generacion procedural de contenido tiene tres objetivos
principales:
\begin{itemize}
    \item Brindar apoyo a los creadores para poder crear contenido mas
    rapidamente.
    \item Utilizar la generacion de contenido para crear juegos que logrean
    reaccionar en tiempo real a las acciones de los usuarios, cosa que en caso
    contrario tendria que encaminarse a escenarios especificos.
    \item Reducir el espacio en memoria tomado por el contenido generado.
\end{itemize}
Una cosa extra que brinda la genracion procedural de contenido es permitir una
mayor creatividad al momento de generar.

\subsection{Áreas de interés de generación procedural}
\label{subsection:PCGAreasOfInterest}

La generación procedural de contenido es un sistema que se enfoca en diferentes
aspectos en el área del entretenimiento especialmente en el área de videojuegos,
estos aspectos pueden o no estar ligados unos con otros y ademas cabe mencionar
que no se enfocan primordialmente en la generación de "objetos" sino mas bien en
la generación de recursos que pueden ser utilizados en el desarrollo de
videojuegos, la generacion procedural se enfoca en seis puntos mostrados en la
imagen \ref{figure:pcg_areas}, cada uno se explica en un apartado iniciando con
el 'Diseño de niveles' en el capitulo \ref{subsection:LevelDesign}, la
generacion de graficos en el capitulo \ref{subsection:Visuals}, la creacion de
audio en el capitulo \ref{subsection:Audio}, creacion de narrativa o historias
en el capitulo \ref{subsection:Narrative}, diseño de reglas y mecanicas en el
capitulo \ref{subsection:rulesandmechanics} y la generacion de juegos completos
explicada en el capitulo \ref{subsection:games}.

\begin{figure}
    \centering
    \includegraphics[width=0.6\textwidth]{img/pcg_areas.png}
    \caption{Areas de PCG en videojuegos}
    \label{figure:pcg_areas}
\end{figure}

\subsection{Diseño de niveles}
\label{subsection:LevelDesign}

La area del desarrollo de niveles es una de las populares en PCG debido a que
los niveles son la parte escencia de un juego debido a que es el area sobre el
cual un jugador puede interactuar, la representacion de estas areas se puede dar
desde imagenes en 2D simplemente con el alto y ancho de los objetos hasta
elementos en 3D que abarquen tambien el grosor de los objetos.

Debido a que los niveles son escencialmente el area principal de interaccion en
un videojuego estos deben de considerar los aspectos funcionalidad y estetica
para que exista una buena interaccion y sea llamativo a los usuarios, de esta
manera se pueden crear areas con tonos oscuros y ambiente tetrico para entregar
un nivel con tematica de terror, la generacion de niveles de juego es un tema
reciente en el ambito de investigacion debido a los elementos que se deben de
tener en cuenta sin embargo dentro del ambito de videojuegos, es un elemento
comunmente utilizado, algunos ejemplos recientes siendo Spelunky, Minecraft,
Disgaea.

Primero tenemos el juego de plataforma Spelunky desarrollado por Derek Yu, el
juego consta de multiples niveles alrededor de cinco diferentes areas, cada area
con un estilo diferente (niveles con hielo, lava, etc.) en este juego los niveles
que recorre el jugador son desarrollados de manera procedural por diferentes
algoritmos dependiendo el area en la que se encuentre el juegador.

El segundo ejemplo es un videojuego desarrollado por Markus Persson llamado
Minecraft, este juego consta de un area estilo caja de arena en donde el jugador
puede realizar las acciones que quiera, las estructuras estan definidas en
manera de bloques con diferentes texturas y propiedades que el jugador puede
utilizar para construir diferentes cosas, la manera en como se utiliza la
generacion de contenido es mediante la generacion de todo un nivel al iniciar el
juego de tal manera que se utiliza una semilla de generacion y se utiliza el
metodo \textit{Perlin Noise} 3D con interpolacion lineal para genrar los biomas,
elementos y entidades que conformaran el nivel, ademas de esto el area inicial
del juego no es la unica area que puede ser recorrida durante la sesion de
juego, sino que el sistema utiliza una "semilla"(seed) que define la manera en
como esta conformado el mapa, mientras mas se va recorriendo del mapa las areas
se van generando acorde a la semilla de manera infinita.

Finalmente tenemos el juego Disgaea desarrollado por Nippon Ichi Software, este
es un juego de estrategia por turnos en donde el objetivo es eliminar a todos
los enemigos en un mapa para continuar al siguiente, la manera en como se
utiliza la generacion de contenido procedural es en una seccion extra del juego
en donde se puede entrar a un arma para completar niveles gradualmente mas
dificiles para darle mas poder a dicha arma, en esta parte los niveles que se
encuentra el jugador son generados proceduralmente tomando en cuenta las alturas
de las partes donde se puede caminar y la colocacion de los enemigos en el mapa
\ref{figure:DisgaeaIW}, para la generacion se toma en cuenta el nivel del arma y
su rareza, mientras mas altos sean estos valores de igual manera los nievels
generados seran mas dificiles.

\begin{figure}
    \centering
    \includegraphics[width=1.0\textwidth]{img/DisgaeaIW.png}
    \caption{Nivel generado en el juego Disgaea}
    \label{figure:DisgaeaIW}
\end{figure}

\subsection{Gráficos}
\label{subsection:Visuals}

La area de desarrollo de graficos se encarga principalmente de generar las
representaciones visuales de los juegos debido a que la mayor parte de los
juegos llevan una parte visual a menos que no se requiera, es necesario generar
una imagen que denote lo que se quiere dar a entender en un juego, de esta
manera se le brinda al jugador un nivel mas de inmersion en la situacion
mediante el uso de paletas de colores o imagenes en pantalla que puedan
representar mejor las situaciones que se presentan.

El ambito de generacion de graficos ah sido uno de los mas explotados debido a
que es posible generar graficos desde simples representaciones de 8 bits como
representaciones photorealisticas de los eventos u objetos presentes en un
juego, tal es el caso explicado en el paper de Risi S. et al.\cite{Risi2012} en
donde utilizan una red de produccion de patrones composisionales (CPPN por sus
siglas en ingles) la cual es una variante de las redes neuronales artificiales
(ANN) regulares, utilizando esta red se busco modificar un circulo de tal manera
que la resultante de tal modificacion creara un patron en forma de una flor, de
igual manera la forma de crear diversidad en los tipos de flores generadas fue
mediante el uso de la neuroevolucion de topologias aumentativas (NEAT) mediante
el cual las redes generadas se alteraban mediante la modificacion de conecciones
entre neuronas y la adicion de nuevos nodos de tal manera que se buscaba un
nivel adecuado de complejidad para las redes que generaron diferentes tipos de
patrones de flores.

Mientras que un segundo paper escrito por Erin J. et al.\cite{Hastings2009} en
el cual presentan un nuevo algoritmo de generacion de contenido llamado
neuroevolucion de topologias aumentativas para generacion de contenido (cgNEAT)
el cual se encarga de generar contendio grafico y contenido del juego deacuerdo
a las preferencias de un jugador mientras el juego esta en ejecucion, para la
evaluacion del contenido generado se desarrollo un juego multijugador en linea
llamado \textit{Galactic Arms Race} en el cual la cgNEAT se encarga de generar
contenido para todos los jugadores y ellos eran quienes proveian la
retroalimentacion del funcionamiento del algoritmo.

\subsection{Audio}
\label{subsection:Audio}

El ambito de generacion de audio en videojuegos se puede considerar como un
punto opcional dependiendo del juego que se este desarrollando sin embargo el
audio juega un papel importante en la experiencia que se le brinda a un jugador
dentro del juego debido a que mediante el uso de componentes sonoros se pueden
influir las emociones que se quieren mostrar en lugares especificos, desde cosas
simples como el uso de sonidos de objetos como escuchar un teclado siendo
utilizado hasta la utilizacion de pistas de audio en areas o momentos clave del
juego para demostrar momentos dramaticos, tristes o de accion. 

Cabe mencionar que es posible utilizar instrumentos musicales o la
implementacion de orquestas para de igual manera generar estos ambientes sin
embargo algunos utilizan este tipo de generacion para crear pistas que sean
capaces de adaptarse a los sucesos que transcurren en el momento para el
concepto de generacion procedural de audio puede ser inclusive el uso de
elementos en un entorno virtual que generen algun sonido cuando son
interactuados \cite{garner2014sonic}, algunos ejemplos de adaptabilida de audio
deacuerdo a los eventos en el momento es el caso de el juego \textit{Fire emblem
Fates}, este es un juego de estrategia por turnos en donde la apatabilidad de
audio se da en transiciones de vista del mapa y combate de unidades, en este
caso el audio de la vista del mapa tiene un cierto nivel de \textit{tempo}
mientras que en momento que se entra a un combate entre dos unidades el
\textit{tempo} del audio cambia mientras se esta en esta escena de combate.

Existen tambien investigaciones academicas de como combinar la generacion de
niveles con la generacion de audio, tal es el caso de \textit{Sonancia}
desarrollado por Phil Lopez et al. \cite{lopes2015sonancia}, en este paper los
autores proponen una metodologia que permitira a un sistema generador de niveles
seleccionar pistas de audio o sonidos que vallan deacuerdo a un tema especifico,
en este caso los autores colocaron una lista de pistas y mediante un generador
crearon un entorno 2-D que simulaba una mansion, el proposito fue el de crear un
juego de terror/suspenso y que las habitaciones tuvieran un audio diferente y
que al acercarse a la division entre una y otra el audio de la habitacion
adyacente fuese subiendo el volumen conforme mas se adentraba y el audio de la
habitacion anterior se dejara de escuchar lentamente, el sistema es capaz de
generar los niveles y asignar los sonidos necesarios sin embargo tiene la
limitante de que se debe de porporcionar el listado de pistas de audio debido a
que aun no tiene la capacidad de generar audio por cuenta propia sin embargo es
otro buen ejemplo de como combinar areas de generacion de contenido.

Otro caso de estudio es el del juego \textit{Audio Surf} desarrolla en 2008 por
Fitterer, la manera en como funciona es que mediante el uso de archivos de audio
de alguna cancion para generar niveles estilo pista de carreras que el usuario
recorre al momento de jugar, como este existen varios otrso juegos que utilizan
mecanicas de PCG para generar los niveles a jugar por los participantes.

Finalmente tenemos \textit{Mezzo} desarrollado por Daniel B. este paper muestra
un programa de computadora del mismo nombre diseñado con el fin de generar
musica de la era romantica que puede ser utilizada en juegos de computadora, el
software fue desarrollado con el proposito de poder darle mas expresividad a
eventos que ocurren en juegos de computadora, esto es mediante el uso de los
elementos presentes en la escena como las partes que conforman la piexa musical,
para esto sus acciones son traducidas como diferentes cantidades de tension
armonica lo cual permite que la pieza musical corresponda al estado del juego
asi como a las acciones que realizan los personajes.

\subsection{Narrativa}
\label{subsection:Narrative}

El ambito de generacion de narrativa en videojuegos se encarga de manera basica
de generar historias que tengan sentido y sean entretenidas, el uso de PCG para
el ambito de narrativa se ah tomado de diferentes angulos, el primero es
sencillamente crear la historia del lugar en donde se desarrollan las cosas,
esto es generar toda la historia desde un punto especifico y asi crear una
"linea" de accion que un jugador debera seguir, mientras que otro de los
aspectos es la generacion de historia que se adapte a las acciones del jugador,
esto es, que a medida que el jugador progrese en un juego realizando acciones
especificas el juego adapte la historia a generar deacuerdo a dichas acciones.

Algunos ejemplos de este ambito de generacion se muestran en el juego
\textit{Facade} y el sistema \textit{Versu}, el primero que se explicara es el
juego de Facade, Facade es un juego desarrollado por Michael Mateas y Andrew
Stern \cite{mateas2003faccade} que intenta crear un juego estilo novela visual
en donde las acciones de el jugadro tengan influencia en los eventos que ocurren
en el juego, el juego introduce al personaje del jugador como un amigo de una
pareja de casados, no se cuenta con un objetivo especifico mas que el de
completar la historia, sin embargo el juego provee de acciones y respuestas que
el personaje puede expresar a la pareja, estas mismas tienen influenciaran
directamente el como se desarrollara la historia, el juego se diseño con el
proposito de que el jugador lo repita multiples veces con el fin de descubrir
como las diferentes decisiones cambiaran el desenlace de la historia. 

Finalmente \textit{Versu} es una aplicacion desarrollado por Richard Evans y
Emily Short \cite{evans2013versu} es juego se encarga de generar simulaciones en
base a lineas de texto que denotan acciones que un conjunto de personajes no
jugables (NPC) realizan en determinados escenarios, el juego permite que a una
persona se le asigne un personaje del juego y controle las acciones que
realizara, mientras que los NPC restantes responderan de manera autonoma
mientras el juego continua, este sistema de generacion de narrativa inicio como
un proyecto academico y despues paso a ser utilizado de manera comercial ademas
de ser la base para el desarrollo de otros juegos de estilo similar.

\subsection{Reglas y Mecanicas}
\label{subsection:rulesandmechanics}

Las reglas y mecanicas de un juego proveeen de un marco de trabajo en el cual el
jugador puede realizar acciones y el como debe de influenciarlas hacia un
objetivo final, las reglas y mecanicas de un juego pueden variar dependiendo del
tipo de juego que se trate, por ejemplo en juegos de estilo plataforma la regla
de terminacion puede ser simplemente llegar hasta cierto punto del nivel
mientras que las macanicas pueden incluir saltar, correr o agacaharse, este
conjunto de reglas permite que el jugador no realize acciones no preevistas por
los programadores y de tal forma limita a un jugador a cierca cantidad de
acciones y consecuentes posibles las cuales debera de aprovechar de diferentes
maneras para continuar.

Mientras que el uso de reglas permite tener un juego mas balanceado algunas
ocasiones el cambiar el conjunto de reglas permite crear diferentes tipos de
juegos, tomando el ejemplo anterior es posible que un juego de plataformas se le
permita a un jugador volar en un nivel lo cual provocaria que las mecanicas se
requieran acomdar a esta nueva habilidad y de igual manera los niveles de un
juego se tengan que reimaginar de tal modo que ahora se pueda utilizar en
ciertas o se coloquen lugares que solo se pueden acceder con esa habilidad por
ejemplo la transicion de reglas y mecanicas entre los juegos Super Mario Bros.
(Nintendo, 1985) y Super Mario World (Nintendo, 1990).

Para la evaluacion de la generacion de mecanicas y reglas de un juego existen
diferentes metodos a utilizar, el primero se basa en el balance de las reglas es
decir en caso de ser un juego para dos o mas jugadores el objetivo de evaluar
seria que todos los jugadores tengan las mismas probabilidades de ganar
utilizando las mecanicas del mismo juego, mientras que otra manera de evaluar
seria mediante la facilidad que tienen las mecanicas de ser aprendidas por los
jugadores.

En cuanto a ejemplos de este tipo de generacion se encuentran algunas
investigaciones academicas tales como la de Ludi \cite{Browne2010}, Ludi es un
sistema desarrollado por Cameron Browne y Frederic Maire, este sistema se
utiliza para evolucionar comandos que definen las reglas de un juego en
particular a crear, el algoritmo evolutivo se encarga de evolucionar las reglas
manteniendolas dentro de un cierto rango de metricas que establecen un buen
patron para juegos de tablero tales como tales como la complejidad del juego o
la simplicidad de las reglas, un juego de mesa diseñado mendiante el uso de este
sistema se llama Yavalath, una vista rapida del juego se puede apreciar en la
figura \ref{figure:Yavalath}, este juego puede tener un minimo de 2 y un maximo
de 3 jugadores, las reglas son que en base a turnos cada jugador coloca una
ficha de su color en cualquier punto no ocupado del tablero, el objetivo es
lograr formar una linea de 4 fichas del mismo color pero la regla es que al
formar una linea de 3 fichas se pierde el juego de manera automatica, de esta
manera el juego se basa en forzar jugadas del oponente que permitan que alguno
tenga mas "control" de los eventos que ocurriran. 

Un segundo caso fue escrito por Togelius et al. \cite{Togelius2008}, este uno de
los primeros ejemplos del uso de computacion evolutiva para la la generacion de
reglas para juegos, en este paper se evoluciona un conjunto de reglas de tal
manera que un conjunto de agentes que participaran en el juego logren tener un
mejor entendimiento de como funcuina el juego, la manera de evaluar que tan
entendibles son las reglas fue a travez de varias sesiones simuladas de los
agentes, en este caso se procuro utilizar reglas sencillas que simularan un
juego estilo pac-man en donde los agentes deberian de colectar circulos para
poder obtener una mejor puntuacion.

Finalmente un ultimo caso es el de Nielsen et al. \cite{Nielsen2015} esos
autores programaron un sistema capaz de representar las reglas de un videojuego
utilizando video game description language en donde juegos estilo 2d se
representan como mapas en base a caracteres y utilizaron ASP para verificar que
los niveles podian ser jugados (se podian cumplir las condiciones de victoria).

\begin{figure}
    \centering
    \includegraphics[width=0.6\textwidth]{img/Yavalath.png}
    \caption{Ejemplo de un juego en progreso en el juego de Yavalath}
    \label{figure:Yavalath}
\end{figure}

\subsection{Juegos}
\label{subsection:games}

Mientras que las facetas de generacion anteriores se han enfocado en aspectos
especificos de juegos mientras que en este aspecto en particular se toma un
enfoque global, esto es desde la generacion de partes visuales hasta la 
generacion de bandas sonoras, esto demuestra como todos los aspectos de un
juegose relacionan unos con otros, un ejemplo de esta relacion seria tomar un
juego en particular y agregar acciones que originalmente no exitian, para poder
agregar estas acciones se debe de considerar el aspecto visual, es decir el como
debera de representarse visualmente tal accion, aspectos de sonidos en caso de
que una accion requiera de algun efecto de sonido que represente tal accion,
utilizando el ejemplo de agregar la mecanica de volar, el aspecto visual se
representaria en un cambio visible en el personaje controlado, por ejemplo,
dibujar alas y que se vea una animacion de movimiento al volar, en el caso del
aspecto del sonido el aleteo o movimiento generado por dichas alas al estar
volando puede tener un sonido unico para identificarlo.

Para este caso existen diferentes maneras de intentar generar juegos nuevos, tal
es el caso de Game-O-Matic \cite{treanor2012game}, Game-O-Matic es una
herramienta de generacion enfocada a la generacion de juegos que representan
ideas, en este sistema se utiliza un sistema de sustantivos unidos mediante
verbos que son ingresados al sistema como datos de entrada, estos datos son
procesados y se logra generar un juego simple estilo arcade que logre
representar un mapa conceptual generado mediante los valores de entrada. 

Otra investigacion realizada por Nelson et al. \cite{Nelson2008}, para este
sistema los autores proponen un sistema parecido al de Game-O-Matic explicado
anteriormente, sin embargo en este caso el sistema esta basado en sprites que
representan entidades del juego, lo que se asigna como valor de entrada es
restricciones que definen la interaccion de los spites en pantalla, las
restricciones son procesadas mediante una red semantica y una base de datos de
lexico, este proceso logra generar juegos que cumplan las restricciones
estipuladas, en este caso los juegos generados tienen mecanicas estilo WarioWare
en donde los juegos tienen una regla de victoria sencilla que hace los juegos
relativamente cortos. 

Finalmente uno de los mejores generadores de juegos lleva en nombre de ANGELINA \cite{cook2011multi} %(Ref ANGELINA num135-137) 
\footnote{Deacuerdo al autor: "A Novel Game-Evolving Labrat I've Named
ANGELINA"}, ANGELINA es un sistema de generacion de juegos diseñado
originalmente en 2011 y que ah recibido modificaciones desde entonces, la base
del generador es el uso de un sistema evolutivo que se encarga de evolucionar
las reglas, los terrenos, los personajes y demas conjuntos con respecto unos de
otros de una manera orquestrada, se encarga no solo de evolucionar la
informacion visual de los componentes sino tambien de asignar sus locaciones en
un campo de trabajo que representa un juego, ademas de esto es capaz de
seleccionar elementos musicales relevantes al tema de los mapas generados o de
la faceta emocional presente en el juego ademas de crear asignarle un nombre a
los juegos que logra crear, inicialmente el sistema era capaz de generar niveles
de plataforma en 2-D pero actualmente es capaz de generar niveles de tematica de
aventura en entornos 3-D.

Los anteriores generadores a pesar de no ser perfectos han permitido dar un gran
paso en la generacion de contenido o mas bien en la generacion de juegos, sin
embargo aun falta controlar aspectos que permitan que las generaciones de tal
contenido logren involucrar todos los aspectos de la generacion de contenido,
esto podria darse mediante el uso de multiples agentes que contantemente evaluen
los aspectos generados pero que al mismo tiempo logren comunicarse unos con
otros y poder evaluar los juegos generados de manera mas amplia.

\section{Jugabilidad}
\label{section:playability}

La jugabilidad es un término utilizado principalmente en el área del análisis y
diseño de videojuegos y es utilizado principalmente como medida de calidad para
la experiencia de un usuario para con el juego mediante las mecánicas o manera
en como se tiene una interacción jugador-videojuego y el diseño como tal del mismo.

De igual manera diferentes autores deinen el concepto como un conjunto de
propiedades que definen la experiencia que logra tener un jugador en un juego
determinado según al sistema de juego que se provee.

\begin{figure}
    \centering
    \includegraphics[width=0.7\textwidth]{img/playabilityv2.png}
    \caption{Puntos de evaluación de jugabilidad}
    \label{figure:playability}
\end{figure}

La figura \ref{figure:playability} muestra un diagrama de areas y conceptos que
califican el nivel de jugabilidad de una aplicacion o juego, estos mismos se
pueden definir como sigue:

\begin{itemize}
    \item Satisfacción: El grado en el que un juego logra agradar o complacer a
    un usuario.
    \item Aprendizaje: El grado con el cual un usuario es capaz de comprender
    las mecanicas de un juego y es capaz de interactuar facilmente con el mismo.
    \item Efectividad: Este demuestra la cantidad de recursos y el tiempo
    utilizados para poder envolver a un usuario y lograr que se divierta.
    \item Inmersión: El grado con el cual se logra envolver a un usuario con los
    eventos que transcuerren en el juego, es decir que tanto se logra hacer que
    se integre o muestre empatia por los eventos del juego.
    \item Motivación: Esta define la capacidad que tiene un juego a motivar a un
    usuario a continuar con los retos o eventos del juego, esto generalmente se
    da por mecanicas que dan un snetimiento de recompensa o simplemente por la
    inmersión lograda.
    \item Emoción: Es el grado con el cual se logra una inmersión en el usuario
    de tal manera que es capaz de reaccionar de manera involuntaria a estimulos
    presentados en el juego, esto puede ser desde una risa por algun evento que
    ocurre hasta saltos involuntarios por susto en juegos de terror.
    \item Socialización: Este es el grado con el cual se logra que un usuario
    entable relaciones con personas que no conoce, en juegos de multijugador
    masivo en linea("Massive-Multiplayer Online" - MMO por sus siglas en ingles)
    existen eventos en los cuales se le pide a los usuarios que contruyan
    equipos para completar niveles en el juego, existen otros juegos de
    multijugador local en los que varias personas compiten para ganar, y en
    ambos casos se puede presentar que de los sucesos presentados se puedan
    crear amistadaes entre jugadores.\cite{sanchez2009playability}
\end{itemize}

Por otro lado existe un conjunto de facetas que engloban varios atributos de la
jugabilidad que permiten relacionar dichos atributos, para estas facetas se
toman los siguientes grupos:

\begin{itemize}
    \item Jugabilidad intrínseca: Engloba el diseño y mecanicas del juego
    enfocandose en las reglas y objetivos establecidos, define el como se
    proyectan estos puntos al jugador.
    \item Jugabilidad mecánica: Se enfoca en parte funcional del juego y en los
    comportamientos de los personajes, se basa en el diseño del motor del juego.
    \item Jugabilidad interactiva: Se enfoca mas en la relacion usuario-juego,
    tomando en cuenta el como se presenta la interfaz de usuario y los controles.
    \item Jugabilidad artística: En este punto se engloban los puntos de
    graficos y sonido, es decir que tan bien es visualmente lo que se presenta y
    el tipo de ambientacion que se crea al combinar efectos visuales, musica y graficos.
    \item Jugabilidad perceptiva: Este punto trata de cuantificar la manera en
    como una persona percive los elementos mostrados en el juego, este es un
    calculo subjetivo.
    \item Jugabilidad interpersonal: Al igual que el punto anterior es un
    concepto subjetivo, sin embargo este se enfoca en las percepciones que se
    generan al jugar en un grupo.
\end{itemize}

\section{Open-Ended Evolution}
\label{section:open-ended_Evolution}

El termino Open-Ended Evolution (OEE) es el nombre que se le da a una variante
del algoritmo de evolucion genetica, este algoritmo difiere de los objetivos
predefinidos de un algoritmo genetico en el cual el objetivo principal de un GA
es el de llegar o aproximarse a un valor o conjunto de valores que logren
solucionar un problema determinado, en el caso de OEE este no es el caso sino
que lo que hace el algoritmo es seguir evolucionando y creando diversidad en la
poblacion cada vez mas compleja no solo dentro de los limites de evolucion de un
elemento sino comenzar a crear nuevas \textit{"especies"} de soluciones.

Una explicacion mas clara es presentada en un paper por T. Taylor et
al.\cite{Taylor2016}, en este paper se define OEE como un sistema capaz de
novedad despues de un punto en la evolucion en donde el estado del grupo o
poblacion no cambia en lugar de converger hacia un estado casi estable.

La manera en como se definira el uso de OEE que se utilizara en este proyecto es
simplemente un proceso de evolucion genetica capaz de encapsular procesos de
evolucion similares capaces de incrementar en numero segun la complejidad a la que
las entidades generadas logren llegar.