\chapter{Conclusiones y trabajo futuro}
\label{chapter:conclusions-and-future-work}

En las siguientes secciones se presentan las conclusiones a las que se llegaron
durante el desarrollo del proyecto de tesis presentado en este documento, sobre
la manera en cómo se implementaron los aspectos propuestos, así como de los
resultados que se lograron obtener mediante el uso de estas propuestas. Así
mismo dentro de la última sección de este capítulo se discuten y proponen
diferentes métodos que pueden ser implementados a futuro para mejorar los
resultados obtenidos, rutas de investigación que se teorizaron podrían funcionar
así como aspectos que requieren de mejorar para un funcionamiento más fluido del
sistema.

\section{Conclusiones}
\label{section:conclusions}

Esta tesis presenta un método novedoso para la generación de niveles del juego de
Angry Birds apoyándose en las mecánicas de la evolución genética y de tomando
aspectos de open-ended evolution. El sistema que se logró generar cuenta con las
capacidades para generar estructuras compuestas utilizando elementos básicos del
juego, además que de que debido a la manera en cómo fue programado es posible
modificar los módulos de clases sin requerir modificaciones enormes en el código
fuente, por ejemplo, es posible realizar modificaciones a las funciones de
selección y mutación que se reflejen de manera rápida en el sistema al realizar
las simulaciones de igual manera el sistema de evaluación se puede mejorar ya
sea cambiando las métricas de evaluación propuestas o agregando nuevas según sea
requerido.

Para fines de este proyecto el sistema fue probado con el simulador de Science
Birds utilizando todos los puntos propuestos en la sección
\ref{chapter:proposed-method} y se obtuvieron resultados como los mostrados en
la sección \ref{chapter:experiments-and-results}, sin embargo, una de las ideas
en el desarrollo de este proyecto es el de permitir la adaptabilidad en el
sentido de que se espera que siempre y cuando un caso de estudio particular es
decir un juego diferente tenga niveles que puedan ser descompuestos hasta las
expresiones mínimas, es decir los bloques de construcción que permitirán generar
un nivel será posible en teoría utilizar este mismo sistema para generar niveles
en otros juegos simplemente ajustando algunas variables que permitan comparar
los compuestos a fin de obtener los resultados esperados.

La implementación mostrada en la sección \ref{chapter:proposed-method} muestra
cómo se explicó anteriormente las ideas que se consideran fueron las mejores
para desarrollar el sistema, sin embargo, aquellas mostradas como propuestas
anteriores demuestran los cambios por los cuales el sistema tuvo que pasar para
poder obtener lo que consideramos como el "mejor" en términos de lo que se
proponía lograr, así como estos mismos métodos podían aproximar resultados a los
obtenidos en el sistema final pueden existir diferentes técnicas que no se
exploraron a detalle que logren obtener resultados iguales o inclusive mejores,
ejemplos como los presentados en la sección \ref{chapter:related-work}
demuestran que diferentes personas lograron tener diferentes métodos de atacar
esta problemática, desde el uso de sistemas inteligentes o sistemas
auto-adaptables, mediante el uso de estas diferentes técnicas se observa una
gran diversidad de resultados, en el caso de este sistema los métodos que se
opina pueden sustituir al algoritmo genético son técnicas basadas en búsquedas
meta-heurísticas debido a la naturaleza del sistema.

Uno de los puntos tratados durante los capítulos \ref{chapter:implementation}
y \ref{chapter:experiments-and-results} es el problema relacionado con el tiempo
de simulación, debido a que los niveles generados requieren ser evaluados dentro
de los parámetros del juego como tal se requiere que el juego se ejecute y
evalué cada nivel uno a la vez, esto provoca que los tiempos de simulación sean
tardados debido a que cada nivel requiere ser simulado durante 10 segundos con
el fin de que las estructuras generadas tengan tiempo de reaccionar bajo la
gravedad, si están mal acomodadas caerán y si están correctamente acomodadas se
mantendrán de pie o balanceadas además de este tiempo de simulación se toma
también en cuenta el tiempo que tarda en escribir los resultados de cada nivel
en sus respectivos archivos eh inclusive el tiempo que el software de simulación
tarda en ser ejecutado, este punto fue uno de los obstáculos más grandes en el
sistema debido a que solo analizar si un cambio funcionaba bien podía tardar
mucho, sin embargo, este tiempo se reduce considerablemente cuando se trabaja en
generaciones diferentes a la inicial debido a que en generaciones subsecuentes
solo se programó que simulara aquellos individuos que fueron generados, gracias
a que este fue un obstáculo que se tuvo que mantener en el sistema se lograron
adaptar partes del sistema para realizar las simulaciones de manera más fluida y
no requerir de volver a ejecutar una simulación para analizar el flujo de los
datos lo cual permitió encontrar errores más rápidamente.

\section{Trabajo futuro}
\label{section:future-work}

Las conclusiones descritas previamente proveen dan un entendimiento de los
obstáculos que por los cuales se tuvo que pasar para poder completar el
proyecto, el problema y posible solución tratados en las conclusiones sobre las
simulaciones generadas es el tiempo desperdiciado debido al uso del software de
simulación, debido a que las simulaciones requieren de este software para poder
obtener los resultados es necesario encontrar una manera en la cual se pueda
estimar mediante el uso de fórmulas las posiciones en las cuales terminaran los
elementos del nivel, considerando que el juego como tal tiene un sistema de
gravedad que provoca que las piezas caigan al suelo debería de ser posible
calcular u obtener el valor de gravedad utilizada para desarrollar un sistema
capaz de calcular el movimiento de las piezas considerando también la
interacción de las piezas unas con otras, así como las resistencias con las que
cuentan.

El ámbito de la evaluación de los individuos es uno de los principales que puede
ser mejorado o modificado, esto es debido a que las funciones que se utilizaron
para la evaluación en esta tesis son las que se consideraron óptimas para la
problemática tratada, de acuerdo a lo que se requiera obtener es posible
modificar la manera en cómo se obtienen los resultados de aptitud de los
individuos, en el caso de lo que se utilizó para el desarrollo del proyecto es
posible modificar los cálculos de los resultados de estabilidad, en este caso lo
que se esperaba obtener es el valor más cercano a 100 para cada individuo en
donde 100 representaba que ninguno de los elementos que conformaban el nivel se
destruyeron o cayeron, sin embargo, es posible utilizar algunas otras funciones o
cálculos para poder obtener estos resultados inclusive cambiar la manera de
interpretar la estabilidad como el uso de cálculos de aceleración promedio o
movimiento angular de las mismas piezas.

Uno de los ámbitos que también requiere ser tratado es el de la estabilidad de
los niveles generados, es decir el sistema logra generar niveles mediante el
acomodo de los compuestos que se tienen registrados, sin embargo, una de las
maneras en cómo el proceso de evaluación, así como el de simulación puede ser
optimizado un poco mas es mediante el uso de reglas de generación o revisión de
patrones en el ordenamiento de los compuestos en los niveles para poder detectar
áreas en donde se pueda estimar que durante la simulación ciertas estructuras
mal acomodadas debido a los componentes básicos que utilizan puedan causar que
los niveles tengan mal rendimiento, es decir, analizar casos en donde una pieza
en particular seguida de otras desfazadas en sus coordenadas \textit{x} o
\textit{y} provoquen que los compuestos se caigan, esto podría ser analizado y
prevenido mediante el uso de diferentes mascaras o en caso de que no sea posible
evitarlo retirar los elementos de la simulación a fin de requerir menos tiempo.

Otro de los factores que pueden ser mejorados es el uso de la lógica detrás de
open-ended evolution, debido a que la consideración de este algoritmo es de
tener una evolución casi infinita y que los niveles del juego se generan una
sola vez al inicio de la evolución realmente no es muy requerido generar nuevos
compuestos de manera continua, pero es posible desarrollar un sistema de
evolución que utilice compuestos resultantes como los mostrados en la sección
\ref{chapter:experiments-and-results} y los integre en una nueva evolución para
combinarlos con componentes básicos e inclusive consigo mismos para generar
nuevos compuestos más complejos que puedan ser integrados a futuras
simulaciones.

Finalmente, un último aspecto que se debe de tomar en cuenta es el número de
experimentos utilizados, como se explicó anteriormente el sistema requería de
mucho tiempo para realizar las simulaciones requeridas por los individuos, esto
inhibió la capacidad de generar una gran cantidad de simulaciones para realizar
comparaciones, sin embargo, con la cantidad actual de resultados obtenidos es
posible validar las capacidades del generador, sin embargo, se requiere realizar
aun más simulaciones utilizando diferentes configuraciones del algoritmo con el
fin de determinar que el sistema es capaz de trabajar correctamente no solo con
el conjunto de configuraciones determinado en la sección de resultados.
