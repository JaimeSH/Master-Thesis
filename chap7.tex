\chapter{Conclusiones y trabajo futuro}
\label{chapter:conclusions-and-future-work}

En las siguientes secciones se presentan las conclusiones a las que se llegaron
durante el desarrollo del proyecto de tesis presentado en este documento, sobre
la manera en como se implementaron los aspectos propuestos asi como de los
resultados que se lograron obtener mediante el uso de estas propuestas. Asi
mismo dentro de la ultima seccion de este capitulo se discuten y proponen
dferentes metodos que pueden ser implementados a futuro para mejorar los
resultados obtenidos, rutas de investigacion que se teorizaron podrian funcionar
asi como aspectos que requieren de mejorar para un funcionamiento mas fluido del
sistema.

\section{Conclusiones}
\label{section:conclusions}

Esta tesis presenta un metodo novedoso para la generacion de nivees del juego de
Angry Birds apoyandose en las mecanicas de la evolucion genetica y de tomando
aspecos de open-ended evolution. El sistema que se logro generar cuenta con las
capacidades para generar estructuras compuestas utilizando elementos basicos del
juego, ademas que de que debido a la manera en como fue programado es posible
modificar los modulos de clases sin requerir modificaciones enormes en el codigo
fuente, por ejemplo es posible realizar modificaciones a las funciones de
seleccion y mutacion que se reflejen de manera rapida en el sistema al realizar
las simulaciones de igual manera el sistema de evaluacion se puede mejorar ya
sea cambiando las metricas de evaluacion propuestas o agregando nuevas segun sea
requerido.

Para fines de este proyecto el sistema fue provado con el simulador de Science
Birds utilizando todos los puntos propuestosen la seccion
\ref{chapter:proposed-method} y se obtuvieron resultados como los mostrados en
la seccion \ref{chapter:experiments-and-results}, sin embargo una de las ideas
en el desarrollo de este proyecto es el de permitir la adaptabilidad en el
sentido de que se espera que siempre y cuando un caso de estudio particular es
decir un juego diferente tenga niveles que puedan ser descompuestos hasta las
expresiones minimas, es decir los bloques de construccion que permitiran generar
un nivel sera posible en teoria utilizar este mismo sistema para generar niveles
en otros juegos simplemente ajustando algunas variables que permitan comparar
los compuestos a fin de obtener los resultados esperados.

La implementacion mostrada en la seccion \ref{chapter:proposed-method} muestra
como se explico anteriormente las ideas que se consideran fueron las mejores
para desarrollar el sistema, sin embargo aquellas mostradas como propuestas
anteriores demuestran los cambios por los cuales el sistema tuvo que pasar para
poder obtener lo que consideramos como el "mejor" en terminos de lo que se
proponia lograr, asi como estos mismos metodos podian aproximar resultados a los
obtenidos en el sistema final pueden existir diferentes tecnicas que no se
exploraron a detalle que logren obtener resultados iguales o inclusive mejores,
ejemplos como los presentados en la seccion \ref{chapter:related-work}
demuestran que diferentes personas lograron tener diferentes metodos de atacar
esta problematica, desde el uso de sistemas inteligentes o sistemas
auto-adaptables, mediante el uso de estas diferentes tecnicas se observa una
gran diversidad de resultados, en el caso de este sistema los metodos que se
opina pueden sustituir al algoritmo genetico son tecnicas basadas en busquedas
meta-heuristicas debido a la naturaleza del sistema.

Uno de los puntos tratadados durante los capitulos \ref{chapter:implementation}
y \ref{chapter:experiments-and-results} es el problema relacionado con el tiempo
de simulacion, debido a que los niveles genrados requieren ser evaluados dentro
de los parametros del juego como tal se requiere que el juego se ejecute y
evalue cada nivel uno a la vez, esto provoca que los tiempos de simulacion sean
tardados debido a que cada nivel requiere ser simulado durante 10 segundos con
el fin de que las estructuras generadas tengan tiempo de reaccionar bajo la
gravedad, si estan mal acomodadas caeran y si estan correctamente acomodadas se
mantendran de pie o balanceadas ademas de este tiempo de simulacion se toma
tambien en cuenta el tiempo que tarda en escribir los resultados de cada nivel
en sus respectivos archivos eh inclusive el tiempo que el software de simulacion
tarda en ser ejecutado, este punto fue uno de los obstaculos mas grandes en el
sistema debido a que solo analizar si un cambio funcionaba bien podia tardar
mucho, sin embargo este tiempo se reduce considerabemente cuando se trabaja en
generaciones diferentes a la inicial debido a que en generaciones subsecuentes
solo se programo que simulara aquellos individuos que fueron generados, gracias
a que este fue un obstaculo que se tuvo que mantener en el sistema se lograron
adaptar partes del sistema para realizar las simulaciones de manera mas fluida y
no requerir de volver a ejetura una simulacion para analizar el flujo de los
datos lo cual permitio encontrar errores mas rapidamente.

\section{Trabajo futuro}
\label{section:future-work}

La conclusiones descritas previamente proveen dan un entendimiento de los
obstaculos que por los cuales se tuvo que pasar para poder completar el
proyecto, el problema y posible solucion tratados en las conclusiones sobre las
simulaciones generadas es el tiempo desperdiciado debido al uso del software de
simulacion, debido a que las simulaciones requieren de este software para poder
obtener los resultados es necesario encontrar una manera en la cual se pueda
estimar mediante el uso de formulas las posiciones en las cuales terminaran los
elementos del nivel, considerando que el juego como tal tiene un sistema de
gravedad que provoca que las piezas caigan al suelo deberia de ser posible
calcular u obtener el valor de gravedad utilizada para desarrollar un sistema
capaz de calcular el movimiento de las piezas considerando tambien la
interaccion de las piezas unas con otras asi como las resistencias con las que
cuentan.

El ambito de la evaluacion de los individuos es uno de los principales que puede
ser mejorado o modificado, esto es debido a que las funciones que se utilizaron
para la evaluacion en esta tesis son las que se consideraron optimas para la
problematica tratada, deacuerdo a lo que se requiera obtener es posible
modificar la manera en como se obtienen los resultados de aptitud de los
individuos, en el caso de lo que se utilizo para el desarrollo del proyecto es
posible modificar los calculos de los resultados de estabilidad, en este caso lo
que se esperaba obtener es el valor mas cercano a 100 para cada individuo en
donde 100 representaba que ninguno de los elementos que conformaban el nivel se
destruyeron o cayeron, sin embargo es posible utilizar alguna otras funciones o
calculos para poder obtener estos resultadoso inclusive cambiar la manera de
interpretar la estabilidad como el uso de calculos de aceleracion promedio o
movimiento angular de las mismas piezas.

Uno de los ambitos que tambien requiere ser tratado es el de la estabilidad de
los niveles generados, es decir el sistema logra generar niveles mediante el
acomodo de los compuestos que se tienen registrados, sin embargo una de las
manera en como el proceso de evalucion asi como el de simulacion puede ser
optimizado un poco mas es mediante el uso de reglas de generacion o revision de
patrones en el ordenamiento de los compuestos en los niveles para poder detectar
areas en donde se pueda estimar que durante la simulacion ciertas estructuras
mal acomodadas debido a los componentes basicos que utilizan puedan causar que
los niveles tengan mal rendimiento, es decir, analizar casos en donde una pieza
en particular seguida de otras desfazadas en sus coordenadas \textit{x} o
\textit{y} provoquen que los compuestos se caigan, esto podria ser analizadao y
prevenido mediante el uso de diferentes mascaras o en caso de que no sea posible
evitarlo retirar los elementos de la simulacion a fin de requerir menos tiempo.

Otro de los factores que pueden ser mejorados es el uso de la logica detras de
open-ended evolution, debido a que la consideracion de este algoritmo es de
tener una evolucion casi infinita y que los niveles del juego se generan una
sola vez al inicio de la evolucion realmente no es muy requerido generar nuevos
compuestos de manera continua pero es posible desarrollar un sistema de
evolucion que utilice compuestos resultantes como los mostrados en la seccion
\ref{chapter:experiments-and-results} y los integre en una nueva evolucion para
combinarlos con componentes basicos e inclusive consigo mismos para generar
nuevos compuestos mas complejos que puedan ser integrados a futuras
simulaciones.

Finalmente un ultimo aspecto que se debe de tomar en cuenta es el numero de
experimentos utilizados, como se explico anterioremente el sistema requeria de
mucho tiempo para realizar las simulaciones requeridas por los individuos, esto
inhibio la capacidad de generar una gran cantidad de simulaciones para realizar
comparaciones, sin embargo con la cantidad actual de resultados obtenidos es
posible validar la capacidades del generador, sin embargo se requiere realizar
aun mas simulaciones utilizando diferentes configuraciones del algoritmo con el
fin de determinar que el sistema es capaz de trabajar correctamente no solo con
el conjunto de configuraciones determinado en la seccion de resultados.